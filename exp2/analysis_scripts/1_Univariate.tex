% Options for packages loaded elsewhere
\PassOptionsToPackage{unicode}{hyperref}
\PassOptionsToPackage{hyphens}{url}
%
\documentclass[
]{article}
\usepackage{amsmath,amssymb}
\usepackage{lmodern}
\usepackage{iftex}
\ifPDFTeX
  \usepackage[T1]{fontenc}
  \usepackage[utf8]{inputenc}
  \usepackage{textcomp} % provide euro and other symbols
\else % if luatex or xetex
  \usepackage{unicode-math}
  \defaultfontfeatures{Scale=MatchLowercase}
  \defaultfontfeatures[\rmfamily]{Ligatures=TeX,Scale=1}
\fi
% Use upquote if available, for straight quotes in verbatim environments
\IfFileExists{upquote.sty}{\usepackage{upquote}}{}
\IfFileExists{microtype.sty}{% use microtype if available
  \usepackage[]{microtype}
  \UseMicrotypeSet[protrusion]{basicmath} % disable protrusion for tt fonts
}{}
\makeatletter
\@ifundefined{KOMAClassName}{% if non-KOMA class
  \IfFileExists{parskip.sty}{%
    \usepackage{parskip}
  }{% else
    \setlength{\parindent}{0pt}
    \setlength{\parskip}{6pt plus 2pt minus 1pt}}
}{% if KOMA class
  \KOMAoptions{parskip=half}}
\makeatother
\usepackage{xcolor}
\IfFileExists{xurl.sty}{\usepackage{xurl}}{} % add URL line breaks if available
\IfFileExists{bookmark.sty}{\usepackage{bookmark}}{\usepackage{hyperref}}
\hypersetup{
  pdftitle={Study 2 Univariate Analysis},
  pdfauthor={Shari Liu},
  hidelinks,
  pdfcreator={LaTeX via pandoc}}
\urlstyle{same} % disable monospaced font for URLs
\usepackage[margin=2cm]{geometry}
\usepackage{color}
\usepackage{fancyvrb}
\newcommand{\VerbBar}{|}
\newcommand{\VERB}{\Verb[commandchars=\\\{\}]}
\DefineVerbatimEnvironment{Highlighting}{Verbatim}{commandchars=\\\{\}}
% Add ',fontsize=\small' for more characters per line
\usepackage{framed}
\definecolor{shadecolor}{RGB}{248,248,248}
\newenvironment{Shaded}{\begin{snugshade}}{\end{snugshade}}
\newcommand{\AlertTok}[1]{\textcolor[rgb]{0.94,0.16,0.16}{#1}}
\newcommand{\AnnotationTok}[1]{\textcolor[rgb]{0.56,0.35,0.01}{\textbf{\textit{#1}}}}
\newcommand{\AttributeTok}[1]{\textcolor[rgb]{0.77,0.63,0.00}{#1}}
\newcommand{\BaseNTok}[1]{\textcolor[rgb]{0.00,0.00,0.81}{#1}}
\newcommand{\BuiltInTok}[1]{#1}
\newcommand{\CharTok}[1]{\textcolor[rgb]{0.31,0.60,0.02}{#1}}
\newcommand{\CommentTok}[1]{\textcolor[rgb]{0.56,0.35,0.01}{\textit{#1}}}
\newcommand{\CommentVarTok}[1]{\textcolor[rgb]{0.56,0.35,0.01}{\textbf{\textit{#1}}}}
\newcommand{\ConstantTok}[1]{\textcolor[rgb]{0.00,0.00,0.00}{#1}}
\newcommand{\ControlFlowTok}[1]{\textcolor[rgb]{0.13,0.29,0.53}{\textbf{#1}}}
\newcommand{\DataTypeTok}[1]{\textcolor[rgb]{0.13,0.29,0.53}{#1}}
\newcommand{\DecValTok}[1]{\textcolor[rgb]{0.00,0.00,0.81}{#1}}
\newcommand{\DocumentationTok}[1]{\textcolor[rgb]{0.56,0.35,0.01}{\textbf{\textit{#1}}}}
\newcommand{\ErrorTok}[1]{\textcolor[rgb]{0.64,0.00,0.00}{\textbf{#1}}}
\newcommand{\ExtensionTok}[1]{#1}
\newcommand{\FloatTok}[1]{\textcolor[rgb]{0.00,0.00,0.81}{#1}}
\newcommand{\FunctionTok}[1]{\textcolor[rgb]{0.00,0.00,0.00}{#1}}
\newcommand{\ImportTok}[1]{#1}
\newcommand{\InformationTok}[1]{\textcolor[rgb]{0.56,0.35,0.01}{\textbf{\textit{#1}}}}
\newcommand{\KeywordTok}[1]{\textcolor[rgb]{0.13,0.29,0.53}{\textbf{#1}}}
\newcommand{\NormalTok}[1]{#1}
\newcommand{\OperatorTok}[1]{\textcolor[rgb]{0.81,0.36,0.00}{\textbf{#1}}}
\newcommand{\OtherTok}[1]{\textcolor[rgb]{0.56,0.35,0.01}{#1}}
\newcommand{\PreprocessorTok}[1]{\textcolor[rgb]{0.56,0.35,0.01}{\textit{#1}}}
\newcommand{\RegionMarkerTok}[1]{#1}
\newcommand{\SpecialCharTok}[1]{\textcolor[rgb]{0.00,0.00,0.00}{#1}}
\newcommand{\SpecialStringTok}[1]{\textcolor[rgb]{0.31,0.60,0.02}{#1}}
\newcommand{\StringTok}[1]{\textcolor[rgb]{0.31,0.60,0.02}{#1}}
\newcommand{\VariableTok}[1]{\textcolor[rgb]{0.00,0.00,0.00}{#1}}
\newcommand{\VerbatimStringTok}[1]{\textcolor[rgb]{0.31,0.60,0.02}{#1}}
\newcommand{\WarningTok}[1]{\textcolor[rgb]{0.56,0.35,0.01}{\textbf{\textit{#1}}}}
\usepackage{longtable,booktabs,array}
\usepackage{calc} % for calculating minipage widths
% Correct order of tables after \paragraph or \subparagraph
\usepackage{etoolbox}
\makeatletter
\patchcmd\longtable{\par}{\if@noskipsec\mbox{}\fi\par}{}{}
\makeatother
% Allow footnotes in longtable head/foot
\IfFileExists{footnotehyper.sty}{\usepackage{footnotehyper}}{\usepackage{footnote}}
\makesavenoteenv{longtable}
\usepackage{graphicx}
\makeatletter
\def\maxwidth{\ifdim\Gin@nat@width>\linewidth\linewidth\else\Gin@nat@width\fi}
\def\maxheight{\ifdim\Gin@nat@height>\textheight\textheight\else\Gin@nat@height\fi}
\makeatother
% Scale images if necessary, so that they will not overflow the page
% margins by default, and it is still possible to overwrite the defaults
% using explicit options in \includegraphics[width, height, ...]{}
\setkeys{Gin}{width=\maxwidth,height=\maxheight,keepaspectratio}
% Set default figure placement to htbp
\makeatletter
\def\fps@figure{htbp}
\makeatother
\setlength{\emergencystretch}{3em} % prevent overfull lines
\providecommand{\tightlist}{%
  \setlength{\itemsep}{0pt}\setlength{\parskip}{0pt}}
\setcounter{secnumdepth}{-\maxdimen} % remove section numbering
\usepackage{fvextra}
\DefineVerbatimEnvironment{Highlighting}{Verbatim}{breaklines,commandchars=\\\{\}}
\ifLuaTeX
  \usepackage{selnolig}  % disable illegal ligatures
\fi

\title{Study 2 Univariate Analysis}
\author{Shari Liu}
\date{2023-01-03}

\begin{document}
\maketitle

\hypertarget{overview}{%
\section{Overview}\label{overview}}

This script visualizes and analyzes the mean amplitude of response
(\texttt{meanbeta}) to expected and unexpected movies involving agents
(psychology movies) and objects (physics movies):

\begin{itemize}
\tightlist
\item
  Part 1: In 8 focal regions (bilateral SMG, STS, insula, inferior
  frontal gyrus, V1, and MT). This is the univariate confirmatory
  analysis for the project.
\item
  Part 2: In a larger set of regions, including 20 putatively
  domain-specific regions (localized using an independent task involving
  Heider \& Simmel - like videos of social vs physical interaction), and
  24 domain-general regions (localized using an independent task
  involving difficult versus easy versions of a spatial working memory
  task), including the regions above. This is the univariate exploratory
  analysis for the project.
\end{itemize}

In Part 3, we also analyze the responses in all of the above regions to
surprising events that involve both psychological and physical
reasoning. This is an additional univariate exploratory analysis.

\begin{Shaded}
\begin{Highlighting}[]
\NormalTok{regioninfo }\OtherTok{\textless{}{-}} \FunctionTok{read.csv}\NormalTok{(}\FunctionTok{here}\NormalTok{(}\StringTok{"input\_data/manyregions\_info.csv"}\NormalTok{)) }
\NormalTok{knitr}\SpecialCharTok{::}\FunctionTok{kable}\NormalTok{(regioninfo }\SpecialCharTok{\%\textgreater{}\%}\NormalTok{ dplyr}\SpecialCharTok{::}\FunctionTok{arrange}\NormalTok{(ROI\_category, focal\_region))}
\end{Highlighting}
\end{Shaded}

\begin{longtable}[]{@{}
  >{\raggedright\arraybackslash}p{(\columnwidth - 6\tabcolsep) * \real{0.4954}}
  >{\raggedright\arraybackslash}p{(\columnwidth - 6\tabcolsep) * \real{0.2661}}
  >{\raggedright\arraybackslash}p{(\columnwidth - 6\tabcolsep) * \real{0.1193}}
  >{\raggedleft\arraybackslash}p{(\columnwidth - 6\tabcolsep) * \real{0.1193}}@{}}
\toprule
\begin{minipage}[b]{\linewidth}\raggedright
ROI\_name
\end{minipage} & \begin{minipage}[b]{\linewidth}\raggedright
ROI\_name\_final
\end{minipage} & \begin{minipage}[b]{\linewidth}\raggedright
ROI\_category
\end{minipage} & \begin{minipage}[b]{\linewidth}\raggedleft
focal\_region
\end{minipage} \\
\midrule
\endhead
lh.MT\_MNI152\_inflated & MT\_L & early\_visual & 0 \\
lh.V1\_MNI152\_inflated & V1\_L & early\_visual & 0 \\
rh.MT\_MNI152\_inflated & MT\_R & early\_visual & 0 \\
rh.V1\_MNI152\_inflated & V1\_R & early\_visual & 0 \\
bilateral.MT\_MNI152\_inflated & MT\_bilateral & early\_visual & 1 \\
bilateral.V1\_MNI152\_inflated & V1\_bilateral & early\_visual & 1 \\
MD\_ROI\_1 & postParietal\_L & MD & 0 \\
MD\_ROI\_10 & medialFrontal\_L & MD & 0 \\
MD\_ROI\_11 & postParietal\_R & MD & 0 \\
MD\_ROI\_12 & midParietal\_R & MD & 0 \\
MD\_ROI\_13 & antParietal\_R & MD & 0 \\
MD\_ROI\_14 & supFrontal\_R & MD & 0 \\
MD\_ROI\_15 & precentral\_A\_preCG\_R & MD & 0 \\
MD\_ROI\_16 & precentral\_B\_IFGop\_R & MD & 0 \\
MD\_ROI\_17 & midFrontal\_R & MD & 0 \\
MD\_ROI\_18 & midFrontalOrb\_R & MD & 0 \\
MD\_ROI\_19 & insula\_R & MD & 0 \\
MD\_ROI\_2 & midParietal\_L & MD & 0 \\
MD\_ROI\_20 & medialFrontal\_R & MD & 0 \\
MD\_ROI\_3 & antParietal\_L & MD & 0 \\
MD\_ROI\_4 & supFrontal\_L & MD & 0 \\
MD\_ROI\_5 & precentral\_A\_preCG\_L & MD & 0 \\
MD\_ROI\_6 & precentral\_B\_IFGop\_L & MD & 0 \\
MD\_ROI\_7 & midFrontal\_L & MD & 0 \\
MD\_ROI\_8 & midFrontalOrb\_L & MD & 0 \\
MD\_ROI\_9 & insula\_L & MD & 0 \\
MD\_ROI\_6\_16\_bilateral & precentral\_B\_IFGop\_bilateral & MD & 1 \\
MD\_ROI\_9\_19\_bilateral & insula\_bilateral & MD & 1 \\
DOTS\_PramodParcel\_phys\_p20\_cluster-2\_inflated &
precentral-supfrontal\_R & physics & 0 \\
DOTS\_PramodParcel\_phys\_p20\_cluster-3\_inflated &
precentral-supfrontal\_L & physics & 0 \\
DOTS\_PramodParcel\_phys\_p20\_cluster-4-left\_inflated & supparietal\_L
& physics & 0 \\
DOTS\_PramodParcel\_phys\_p20\_cluster-4-right\_inflated &
supparietal\_R & physics & 0 \\
DOTS\_VOE234\_phys\_p20\_cluster-5-left\_inflated & vismedial\_L &
physics & 0 \\
DOTS\_VOE234\_phys\_p20\_cluster-5-right\_inflated & vismedial\_R &
physics & 0 \\
DOTS\_VOE234\_phys\_p20\_cluster-4-left\_inflated\_notSTS &
supramarginal\_L & physics & 1 \\
DOTS\_VOE234\_phys\_p20\_cluster-4-right\_inflated\_notSTS &
supramarginal\_R & physics & 1 \\
DOTS\_VOE234\_soc\_p20\_cluster-14-21\_inflated & vislateralventral\_R &
psychology & 0 \\
DOTS\_VOE234\_soc\_p20\_cluster-20\_inflated & vislateralventral\_L &
psychology & 0 \\
DOTS\_VOE234\_soc\_p20\_cluster-6-10-11-left\_inflated & MPFC\_L &
psychology & 0 \\
DOTS\_VOE234\_soc\_p20\_cluster-6-10-11-right\_inflated & MPFC\_R &
psychology & 0 \\
DOTS\_VOE234\_soc\_p20\_cluster-7\_inflated & parahip-gyrus\_L &
psychology & 0 \\
DOTS\_VOE234\_soc\_p20\_cluster-8\_inflated & parahip-gyrus\_R &
psychology & 0 \\
DOTS\_VOE234\_soc\_p20\_cluster-9-12-17-18-left\_inflated &
supinffrontal\_L & psychology & 0 \\
DOTS\_VOE234\_soc\_p20\_cluster-9-12-17-18-right\_inflated &
supinffrontal\_R & psychology & 0 \\
DOTS\_VOE234\_soc\_p20\_cluster-16-19\_inflated & superiortemporal\_R &
psychology & 1 \\
DOTS\_VOE234\_soc\_p20\_cluster-22\_inflated & superiortemporal\_L &
psychology & 1 \\
\bottomrule
\end{longtable}

\begin{figure}
\centering
\includegraphics{input_data/MD_parcels.jpg}
\caption{MD Regions from EvLab.}
\end{figure}

\begin{Shaded}
\begin{Highlighting}[]
\NormalTok{univariate\_data }\OtherTok{\textless{}{-}}
  \FunctionTok{readRDS}\NormalTok{(}\FunctionTok{here}\NormalTok{(}\StringTok{"outputs"}\NormalTok{, }\StringTok{"study2\_univariate\_data.Rds"}\NormalTok{))}

\NormalTok{univariate\_summary\_domain }\OtherTok{\textless{}{-}}
  \FunctionTok{readRDS}\NormalTok{(}\FunctionTok{here}\NormalTok{(}\StringTok{"outputs"}\NormalTok{, }\StringTok{"study2\_univariate\_summary\_domain.Rds"}\NormalTok{))}
\end{Highlighting}
\end{Shaded}

\hypertarget{part-1-focal-regions---confirmatory-analysis}{%
\section{PART 1: Focal regions - confirmatory
analysis}\label{part-1-focal-regions---confirmatory-analysis}}

\begin{Shaded}
\begin{Highlighting}[]
\NormalTok{focal\_data }\OtherTok{\textless{}{-}}\NormalTok{ univariate\_data }\SpecialCharTok{\%\textgreater{}\%}
  \FunctionTok{filter}\NormalTok{(focal\_region }\SpecialCharTok{==} \DecValTok{1}\NormalTok{)}

\NormalTok{focal\_summary\_domain }\OtherTok{\textless{}{-}}\NormalTok{ univariate\_summary\_domain }\SpecialCharTok{\%\textgreater{}\%}
  \FunctionTok{filter}\NormalTok{(focal\_region }\SpecialCharTok{==} \DecValTok{1}\NormalTok{)}

\NormalTok{domain\_specific\_regions }\OtherTok{\textless{}{-}} \FunctionTok{c}\NormalTok{(}\StringTok{"physics"}\NormalTok{, }\StringTok{"psychology"}\NormalTok{)}
\NormalTok{domain\_general\_regions }\OtherTok{\textless{}{-}} \FunctionTok{c}\NormalTok{(}\StringTok{"early\_visual"}\NormalTok{, }\StringTok{"MD"}\NormalTok{)}

\NormalTok{regions }\OtherTok{\textless{}{-}} \FunctionTok{levels}\NormalTok{(}\FunctionTok{as.factor}\NormalTok{(focal\_data}\SpecialCharTok{$}\NormalTok{ROI\_name\_final))}
\end{Highlighting}
\end{Shaded}

\hypertarget{plots}{%
\subsection{Plots}\label{plots}}

First, here are plots of betas, run by run, to physics and psychology
events (familiarization, expected, unexpected).

\begin{Shaded}
\begin{Highlighting}[]
\FunctionTok{plot\_univar\_runbyrun}\NormalTok{(}\StringTok{"superiortemporal\_L"}\NormalTok{)}
\end{Highlighting}
\end{Shaded}

\begin{verbatim}
## Warning: Using alpha for a discrete variable is not advised.
\end{verbatim}

\includegraphics[width=1\linewidth]{1_Univariate_files/figure-latex/unnamed-chunk-5-1}

\begin{Shaded}
\begin{Highlighting}[]
\FunctionTok{plot\_univar\_runbyrun}\NormalTok{(}\StringTok{"superiortemporal\_R"}\NormalTok{)}
\end{Highlighting}
\end{Shaded}

\begin{verbatim}
## Warning: Using alpha for a discrete variable is not advised.
\end{verbatim}

\includegraphics[width=1\linewidth]{1_Univariate_files/figure-latex/unnamed-chunk-5-2}

\begin{Shaded}
\begin{Highlighting}[]
\FunctionTok{plot\_univar\_runbyrun}\NormalTok{(}\StringTok{"supramarginal\_L"}\NormalTok{)}
\end{Highlighting}
\end{Shaded}

\begin{verbatim}
## Warning: Using alpha for a discrete variable is not advised.
\end{verbatim}

\includegraphics[width=1\linewidth]{1_Univariate_files/figure-latex/unnamed-chunk-5-3}

\begin{Shaded}
\begin{Highlighting}[]
\FunctionTok{plot\_univar\_runbyrun}\NormalTok{(}\StringTok{"supramarginal\_R"}\NormalTok{)}
\end{Highlighting}
\end{Shaded}

\begin{verbatim}
## Warning: Using alpha for a discrete variable is not advised.
\end{verbatim}

\includegraphics[width=1\linewidth]{1_Univariate_files/figure-latex/unnamed-chunk-5-4}

\begin{Shaded}
\begin{Highlighting}[]
\FunctionTok{plot\_univar\_runbyrun}\NormalTok{(}\StringTok{"precentral\_B\_IFGop\_bilateral"}\NormalTok{)}
\end{Highlighting}
\end{Shaded}

\begin{verbatim}
## Warning: Using alpha for a discrete variable is not advised.
\end{verbatim}

\includegraphics[width=1\linewidth]{1_Univariate_files/figure-latex/unnamed-chunk-5-5}

\begin{Shaded}
\begin{Highlighting}[]
\FunctionTok{plot\_univar\_runbyrun}\NormalTok{(}\StringTok{"insula\_bilateral"}\NormalTok{)}
\end{Highlighting}
\end{Shaded}

\begin{verbatim}
## Warning: Using alpha for a discrete variable is not advised.
\end{verbatim}

\includegraphics[width=1\linewidth]{1_Univariate_files/figure-latex/unnamed-chunk-5-6}

\begin{Shaded}
\begin{Highlighting}[]
\FunctionTok{plot\_univar\_runbyrun}\NormalTok{(}\StringTok{"V1\_bilateral"}\NormalTok{)}
\end{Highlighting}
\end{Shaded}

\begin{verbatim}
## Warning: Using alpha for a discrete variable is not advised.
\end{verbatim}

\includegraphics[width=1\linewidth]{1_Univariate_files/figure-latex/unnamed-chunk-5-7}

\begin{Shaded}
\begin{Highlighting}[]
\FunctionTok{plot\_univar\_runbyrun}\NormalTok{(}\StringTok{"MT\_bilateral"}\NormalTok{)}
\end{Highlighting}
\end{Shaded}

\begin{verbatim}
## Warning: Using alpha for a discrete variable is not advised.
\end{verbatim}

\includegraphics[width=1\linewidth]{1_Univariate_files/figure-latex/unnamed-chunk-5-8}

And here are the same plots given second level maps of the same data.
Clearly there is heterogeneity across runs\ldots!

\begin{Shaded}
\begin{Highlighting}[]
\NormalTok{plot\_univar\_allruns }\OtherTok{\textless{}{-}} \ControlFlowTok{function}\NormalTok{(region) \{}
\NormalTok{  plotobject }\OtherTok{\textless{}{-}}
    \FunctionTok{ggplot}\NormalTok{(}
      \AttributeTok{data =}\NormalTok{ focal\_summary\_domain }\SpecialCharTok{\%\textgreater{}\%} \FunctionTok{filter}\NormalTok{(}
        \FunctionTok{str\_detect}\NormalTok{(ROI\_name\_final, region),}
\NormalTok{        extracted\_run\_number }\SpecialCharTok{==} \StringTok{"all\_runs"}\NormalTok{,}
\NormalTok{        domain }\SpecialCharTok{!=} \StringTok{"both"}
\NormalTok{      ),}
      \FunctionTok{aes}\NormalTok{(}\AttributeTok{x =}\NormalTok{ event, }\AttributeTok{y =}\NormalTok{ meanbeta, }\AttributeTok{fill =}\NormalTok{ domain)}
\NormalTok{    ) }\SpecialCharTok{+}
    \FunctionTok{geom\_bar}\NormalTok{(}\AttributeTok{stat =} \StringTok{"identity"}\NormalTok{, }\FunctionTok{aes}\NormalTok{(}\AttributeTok{alpha =}\NormalTok{ event), }\AttributeTok{colour =} \StringTok{"black"}\NormalTok{) }\SpecialCharTok{+}
    \FunctionTok{geom\_errorbar}\NormalTok{(}
      \FunctionTok{aes}\NormalTok{(}\AttributeTok{ymin =}\NormalTok{ meanbeta }\SpecialCharTok{{-}}\NormalTok{ se, }\AttributeTok{ymax =}\NormalTok{ meanbeta }\SpecialCharTok{+}\NormalTok{ se),}
      \AttributeTok{position =} \FunctionTok{position\_dodge}\NormalTok{(}\AttributeTok{width =}\NormalTok{ .}\DecValTok{9}\NormalTok{),}
      \AttributeTok{width =}\NormalTok{ .}\DecValTok{2}\NormalTok{,}
      \AttributeTok{colour =} \StringTok{"black"}
\NormalTok{    ) }\SpecialCharTok{+}
    \FunctionTok{geom\_point}\NormalTok{(}\AttributeTok{data =}\NormalTok{ focal\_data }\SpecialCharTok{\%\textgreater{}\%} \FunctionTok{filter}\NormalTok{(}\FunctionTok{str\_detect}\NormalTok{(ROI\_name\_final, region), extracted\_run\_number }\SpecialCharTok{==} \StringTok{"all\_runs"}\NormalTok{, domain }\SpecialCharTok{!=} \StringTok{"both"}
\NormalTok{),}
               \AttributeTok{alpha =}\NormalTok{ .}\DecValTok{1}\NormalTok{) }\SpecialCharTok{+}
    \FunctionTok{geom\_line}\NormalTok{(}\AttributeTok{data =}\NormalTok{ focal\_data }\SpecialCharTok{\%\textgreater{}\%} \FunctionTok{filter}\NormalTok{(}\FunctionTok{str\_detect}\NormalTok{(ROI\_name\_final, region),                                                extracted\_run\_number }\SpecialCharTok{==} \StringTok{"all\_runs"}\NormalTok{, domain }\SpecialCharTok{!=} \StringTok{"both"}
\NormalTok{),}
              \FunctionTok{aes}\NormalTok{(}\AttributeTok{group =}
\NormalTok{                    subjectID),}
              \AttributeTok{alpha =}\NormalTok{ .}\DecValTok{1}\NormalTok{) }\SpecialCharTok{+}
    \FunctionTok{theme\_cowplot}\NormalTok{(}\DecValTok{10}\NormalTok{) }\SpecialCharTok{+}
    \CommentTok{\# facet\_wrap(\textasciitilde{} ROI\_name\_final + domain, nrow = 1) +}
    \FunctionTok{facet\_wrap}\NormalTok{( }\SpecialCharTok{\textasciitilde{}}\NormalTok{ extracted\_run\_number }\SpecialCharTok{+}\NormalTok{ ROI\_name\_final }\SpecialCharTok{+}\NormalTok{ domain, }\AttributeTok{nrow =} \DecValTok{1}\NormalTok{) }\SpecialCharTok{+}
    \FunctionTok{scale\_fill\_manual}\NormalTok{(}\AttributeTok{values =} \FunctionTok{c}\NormalTok{(}\StringTok{"\#00AFBB"}\NormalTok{, }\StringTok{"\#FC4E07"}\NormalTok{)) }\SpecialCharTok{+}
    \FunctionTok{ylab}\NormalTok{(}\StringTok{"Average beta"}\NormalTok{) }\SpecialCharTok{+}
    \FunctionTok{xlab}\NormalTok{(}\StringTok{"Event"}\NormalTok{) }\SpecialCharTok{+}
    \FunctionTok{theme}\NormalTok{(}\AttributeTok{axis.text.x =} \FunctionTok{element\_text}\NormalTok{(}
      \AttributeTok{angle =} \DecValTok{90}\NormalTok{,}
      \AttributeTok{vjust =} \FloatTok{0.5}\NormalTok{,}
      \AttributeTok{hjust =} \DecValTok{1}
\NormalTok{    )) }\SpecialCharTok{+}
    \FunctionTok{ggtitle}\NormalTok{(}\FunctionTok{paste0}\NormalTok{(}\StringTok{"ROI:"}\NormalTok{, region)) }
  
\NormalTok{  plotobject}
\NormalTok{\}}
\end{Highlighting}
\end{Shaded}

\begin{Shaded}
\begin{Highlighting}[]
\FunctionTok{plot\_univar\_allruns}\NormalTok{(}\StringTok{"superiortemporal\_L"}\NormalTok{)}
\end{Highlighting}
\end{Shaded}

\begin{verbatim}
## Warning: Using alpha for a discrete variable is not advised.
\end{verbatim}

\includegraphics[width=1\linewidth]{1_Univariate_files/figure-latex/unnamed-chunk-7-1}

\begin{Shaded}
\begin{Highlighting}[]
\FunctionTok{plot\_univar\_allruns}\NormalTok{(}\StringTok{"superiortemporal\_R"}\NormalTok{)}
\end{Highlighting}
\end{Shaded}

\begin{verbatim}
## Warning: Using alpha for a discrete variable is not advised.
\end{verbatim}

\includegraphics[width=1\linewidth]{1_Univariate_files/figure-latex/unnamed-chunk-7-2}

\begin{Shaded}
\begin{Highlighting}[]
\FunctionTok{plot\_univar\_allruns}\NormalTok{(}\StringTok{"supramarginal\_L"}\NormalTok{)}
\end{Highlighting}
\end{Shaded}

\begin{verbatim}
## Warning: Using alpha for a discrete variable is not advised.
\end{verbatim}

\includegraphics[width=1\linewidth]{1_Univariate_files/figure-latex/unnamed-chunk-7-3}

\begin{Shaded}
\begin{Highlighting}[]
\FunctionTok{plot\_univar\_allruns}\NormalTok{(}\StringTok{"supramarginal\_R"}\NormalTok{)}
\end{Highlighting}
\end{Shaded}

\begin{verbatim}
## Warning: Using alpha for a discrete variable is not advised.
\end{verbatim}

\includegraphics[width=1\linewidth]{1_Univariate_files/figure-latex/unnamed-chunk-7-4}

\begin{Shaded}
\begin{Highlighting}[]
\FunctionTok{plot\_univar\_allruns}\NormalTok{(}\StringTok{"precentral\_B\_IFGop\_bilateral"}\NormalTok{)}
\end{Highlighting}
\end{Shaded}

\begin{verbatim}
## Warning: Using alpha for a discrete variable is not advised.
\end{verbatim}

\includegraphics[width=1\linewidth]{1_Univariate_files/figure-latex/unnamed-chunk-7-5}

\begin{Shaded}
\begin{Highlighting}[]
\FunctionTok{plot\_univar\_allruns}\NormalTok{(}\StringTok{"insula\_bilateral"}\NormalTok{)}
\end{Highlighting}
\end{Shaded}

\begin{verbatim}
## Warning: Using alpha for a discrete variable is not advised.
\end{verbatim}

\includegraphics[width=1\linewidth]{1_Univariate_files/figure-latex/unnamed-chunk-7-6}

\begin{Shaded}
\begin{Highlighting}[]
\FunctionTok{plot\_univar\_allruns}\NormalTok{(}\StringTok{"V1\_bilateral"}\NormalTok{)}
\end{Highlighting}
\end{Shaded}

\begin{verbatim}
## Warning: Using alpha for a discrete variable is not advised.
\end{verbatim}

\includegraphics[width=1\linewidth]{1_Univariate_files/figure-latex/unnamed-chunk-7-7}

\begin{Shaded}
\begin{Highlighting}[]
\FunctionTok{plot\_univar\_allruns}\NormalTok{(}\StringTok{"MT\_bilateral"}\NormalTok{)}
\end{Highlighting}
\end{Shaded}

\begin{verbatim}
## Warning: Using alpha for a discrete variable is not advised.
\end{verbatim}

\includegraphics[width=1\linewidth]{1_Univariate_files/figure-latex/unnamed-chunk-7-8}

\hypertarget{check-whether-we-should-restrict-to-first-2-runs}{%
\subsection{Check whether we should restrict to first 2
runs}\label{check-whether-we-should-restrict-to-first-2-runs}}

Following the preregistration plan, we check whether we are able to take
the second level cope values (averaging across 4 runs) as the main
dependent measure, or whether we need to restrict the analysis to just
the first 2 runs.

First we fit a model including an interaction between run number and
event.

\begin{Shaded}
\begin{Highlighting}[]
\FunctionTok{options}\NormalTok{(}\AttributeTok{contrasts =} \FunctionTok{c}\NormalTok{(}\StringTok{"contr.sum"}\NormalTok{, }\StringTok{"contr.poly"}\NormalTok{))}

\NormalTok{run\_data }\OtherTok{\textless{}{-}}\NormalTok{ focal\_data }\SpecialCharTok{\%\textgreater{}\%}
  \FunctionTok{filter}\NormalTok{(event }\SpecialCharTok{!=} \StringTok{"fam"}\NormalTok{,}
\NormalTok{         top\_n\_voxels }\SpecialCharTok{==} \StringTok{"100"}\NormalTok{,}
\NormalTok{         domain }\SpecialCharTok{!=} \StringTok{"both"}\NormalTok{,}
         \CommentTok{\# extracted\_run\_number == "all\_runs") \%\textgreater{}\%}
         \FunctionTok{str\_length}\NormalTok{(extracted\_run\_number) }\SpecialCharTok{==} \DecValTok{4}\NormalTok{) }\SpecialCharTok{\%\textgreater{}\%}
  \FunctionTok{filter}\NormalTok{(}\SpecialCharTok{!}\FunctionTok{str\_detect}\NormalTok{(ROI\_name\_final, }\StringTok{"V1"}\NormalTok{) }\SpecialCharTok{|}
           \SpecialCharTok{!}\FunctionTok{str\_detect}\NormalTok{(ROI\_name\_final, }\StringTok{"MT"}\NormalTok{))}

\NormalTok{run\_data }\SpecialCharTok{\%\textgreater{}\%}
  \FunctionTok{group\_by}\NormalTok{(subjectID, ROI\_name\_final, extracted\_run\_number) }\SpecialCharTok{\%\textgreater{}\%}
  \FunctionTok{summarise}\NormalTok{(}\AttributeTok{n =} \FunctionTok{n}\NormalTok{())}
\end{Highlighting}
\end{Shaded}

\begin{verbatim}
## `summarise()` has grouped output by 'subjectID', 'ROI_name_final'. You can
## override using the `.groups` argument.
\end{verbatim}

\begin{verbatim}
## # A tibble: 1,016 x 4
## # Groups:   subjectID, ROI_name_final [256]
##    subjectID       ROI_name_final               extracted_run_number     n
##    <chr>           <chr>                        <fct>                <int>
##  1 sub-SAXNES2s001 insula_bilateral             run1                     4
##  2 sub-SAXNES2s001 insula_bilateral             run2                     4
##  3 sub-SAXNES2s001 insula_bilateral             run3                     4
##  4 sub-SAXNES2s001 insula_bilateral             run4                     4
##  5 sub-SAXNES2s001 MT_bilateral                 run1                     4
##  6 sub-SAXNES2s001 MT_bilateral                 run2                     4
##  7 sub-SAXNES2s001 MT_bilateral                 run3                     4
##  8 sub-SAXNES2s001 MT_bilateral                 run4                     4
##  9 sub-SAXNES2s001 precentral_B_IFGop_bilateral run1                     4
## 10 sub-SAXNES2s001 precentral_B_IFGop_bilateral run2                     4
## # ... with 1,006 more rows
\end{verbatim}

\begin{Shaded}
\begin{Highlighting}[]
\NormalTok{run\_data}\SpecialCharTok{$}\NormalTok{fixation\_position }\OtherTok{\textless{}{-}} \FunctionTok{as.factor}\NormalTok{(run\_data}\SpecialCharTok{$}\NormalTok{fixation\_position)}

\NormalTok{run\_data}\SpecialCharTok{$}\NormalTok{event }\OtherTok{\textless{}{-}} \FunctionTok{relevel}\NormalTok{(run\_data}\SpecialCharTok{$}\NormalTok{event, }\AttributeTok{ref =} \StringTok{"unexp"}\NormalTok{)}

\NormalTok{checkmodel }\OtherTok{\textless{}{-}} \FunctionTok{lmer}\NormalTok{(}
  \AttributeTok{data =}\NormalTok{ run\_data,}
  \AttributeTok{formula =}\NormalTok{ meanbeta }\SpecialCharTok{\textasciitilde{}}\NormalTok{ event }\SpecialCharTok{*}\NormalTok{ extracted\_run\_number }\SpecialCharTok{+}\NormalTok{ (}\DecValTok{1}\SpecialCharTok{|}\NormalTok{extracted\_run\_number) }\SpecialCharTok{+}\NormalTok{ (}\DecValTok{1}\SpecialCharTok{|}\NormalTok{subjectID) }\SpecialCharTok{+}\NormalTok{ (}\DecValTok{1}\SpecialCharTok{|}\NormalTok{ROI\_name\_final)}
\NormalTok{)}

\NormalTok{checkmodel }\SpecialCharTok{\%\textgreater{}\%}
 \FunctionTok{apa\_print}\NormalTok{() }\SpecialCharTok{\%\textgreater{}\%}
 \FunctionTok{apa\_table}\NormalTok{()}
\end{Highlighting}
\end{Shaded}

\begin{table}[tbp]

\begin{center}
\begin{threeparttable}

\caption{\label{tab:unnamed-chunk-8}}

\begin{tabular}{llllll}
\toprule
Term & \multicolumn{1}{c}{$\hat{\beta}$} & \multicolumn{1}{c}{95\% CI} & \multicolumn{1}{c}{$t$} & \multicolumn{1}{c}{$\mathit{df}$} & \multicolumn{1}{c}{$p$}\\
\midrule
Intercept & 3.02 & {}[0.91, 5.12] & 2.81 & 0.00 & > .999\\
Event1 & 0.07 & {}[-0.01, 0.14] & 1.71 & 4,018.01 & .087\\
Extracted run number1 & 0.29 & {}[-0.49, 1.06] & 0.73 & 0.00 & > .999\\
Extracted run number2 & 0.13 & {}[-0.65, 0.90] & 0.32 & 0.00 & > .999\\
Extracted run number3 & -0.11 & {}[-0.88, 0.67] & -0.27 & 0.00 & > .999\\
Event1 $\times$ Extracted run number1 & 0.03 & {}[-0.10, 0.16] & 0.42 & 4,018.01 & .672\\
Event1 $\times$ Extracted run number2 & 0.05 & {}[-0.08, 0.18] & 0.74 & 4,018.01 & .462\\
Event1 $\times$ Extracted run number3 & -0.06 & {}[-0.19, 0.07] & -0.89 & 4,018.01 & .375\\
\bottomrule
\end{tabular}

\end{threeparttable}
\end{center}

\end{table}

\begin{Shaded}
\begin{Highlighting}[]
\CommentTok{\# summary(checkmodel)}
\FunctionTok{lsmeans}\NormalTok{(checkmodel, pairwise }\SpecialCharTok{\textasciitilde{}}\NormalTok{ event }\SpecialCharTok{|}\NormalTok{ extracted\_run\_number)}\SpecialCharTok{$}\NormalTok{contrasts}
\end{Highlighting}
\end{Shaded}

\begin{verbatim}
## Note: D.f. calculations have been disabled because the number of observations exceeds 3000.
## To enable adjustments, add the argument 'pbkrtest.limit = 4064' (or larger)
## [or, globally, 'set emm_options(pbkrtest.limit = 4064)' or larger];
## but be warned that this may result in large computation time and memory use.
## Note: D.f. calculations have been disabled because the number of observations exceeds 3000.
## To enable adjustments, add the argument 'lmerTest.limit = 4064' (or larger)
## [or, globally, 'set emm_options(lmerTest.limit = 4064)' or larger];
## but be warned that this may result in large computation time and memory use.
\end{verbatim}

\begin{verbatim}
## extracted_run_number = run1:
##  contrast    estimate    SE  df z.ratio p.value
##  unexp - exp   0.1866 0.152 Inf 1.225   0.2210 
## 
## extracted_run_number = run2:
##  contrast    estimate    SE  df z.ratio p.value
##  unexp - exp   0.2279 0.152 Inf 1.497   0.1350 
## 
## extracted_run_number = run3:
##  contrast    estimate    SE  df z.ratio p.value
##  unexp - exp   0.0124 0.155 Inf 0.080   0.9360 
## 
## extracted_run_number = run4:
##  contrast    estimate    SE  df z.ratio p.value
##  unexp - exp   0.0959 0.152 Inf 0.630   0.5290 
## 
## Degrees-of-freedom method: asymptotic
\end{verbatim}

We then compare this model to a simpler model without this interaction

\begin{Shaded}
\begin{Highlighting}[]
\NormalTok{checkmodel\_simpler }\OtherTok{\textless{}{-}} \FunctionTok{lmer}\NormalTok{(}
  \AttributeTok{data =}\NormalTok{ run\_data,}
  \AttributeTok{formula =}\NormalTok{ meanbeta }\SpecialCharTok{\textasciitilde{}}\NormalTok{ event }\SpecialCharTok{+}\NormalTok{ extracted\_run\_number }\SpecialCharTok{+}\NormalTok{ (}\DecValTok{1}\SpecialCharTok{|}\NormalTok{extracted\_run\_number) }\SpecialCharTok{+}\NormalTok{ (}\DecValTok{1}\SpecialCharTok{|}\NormalTok{subjectID) }\SpecialCharTok{+}\NormalTok{ (}\DecValTok{1}\SpecialCharTok{|}\NormalTok{ROI\_name\_final)}
\NormalTok{)}
\end{Highlighting}
\end{Shaded}

\begin{verbatim}
## Warning in checkConv(attr(opt, "derivs"), opt$par, ctrl = control$checkConv, :
## unable to evaluate scaled gradient
\end{verbatim}

\begin{verbatim}
## Warning in checkConv(attr(opt, "derivs"), opt$par, ctrl = control$checkConv, :
## Model failed to converge: degenerate Hessian with 1 negative eigenvalues
\end{verbatim}

\begin{verbatim}
## Warning: Model failed to converge with 1 negative eigenvalue: -3.8e-06
\end{verbatim}

\begin{Shaded}
\begin{Highlighting}[]
\CommentTok{\# summary(checkmodel\_simpler)}
\NormalTok{checkmodel\_simpler }\SpecialCharTok{\%\textgreater{}\%}
  \FunctionTok{apa\_print}\NormalTok{() }\SpecialCharTok{\%\textgreater{}\%}
  \FunctionTok{apa\_table}\NormalTok{()}
\end{Highlighting}
\end{Shaded}

\begin{table}[tbp]

\begin{center}
\begin{threeparttable}

\caption{\label{tab:unnamed-chunk-9}}

\begin{tabular}{llllll}
\toprule
Term & \multicolumn{1}{c}{$\hat{\beta}$} & \multicolumn{1}{c}{95\% CI} & \multicolumn{1}{c}{$t$} & \multicolumn{1}{c}{$\mathit{df}$} & \multicolumn{1}{c}{$p$}\\
\midrule
Intercept & 3.02 & {}[0.91, 5.12] & 2.81 & 8.19 & .022\\
Event1 & 0.07 & {}[-0.01, 0.14] & 1.72 & 4,021.01 & .085\\
Extracted run number1 & 0.29 & {}[-0.48, 1.06] & 0.73 & 4,021.02 & .466\\
Extracted run number2 & 0.13 & {}[-0.64, 0.90] & 0.32 & 4,021.02 & .747\\
Extracted run number3 & -0.11 & {}[-0.88, 0.66] & -0.28 & 4,021.10 & .783\\
\bottomrule
\end{tabular}

\end{threeparttable}
\end{center}

\end{table}

\begin{Shaded}
\begin{Highlighting}[]
\FunctionTok{anova}\NormalTok{(checkmodel, checkmodel\_simpler)}
\end{Highlighting}
\end{Shaded}

\begin{verbatim}
## refitting model(s) with ML (instead of REML)
\end{verbatim}

\begin{verbatim}
## Data: run_data
## Models:
## checkmodel_simpler: meanbeta ~ event + extracted_run_number + (1 | extracted_run_number) + 
## checkmodel_simpler:     (1 | subjectID) + (1 | ROI_name_final)
## checkmodel: meanbeta ~ event * extracted_run_number + (1 | extracted_run_number) + 
## checkmodel:     (1 | subjectID) + (1 | ROI_name_final)
##                    npar   AIC   BIC logLik deviance Chisq Df Pr(>Chisq)
## checkmodel_simpler    9 18938 18995  -9460    18920                    
## checkmodel           12 18943 19018  -9459    18919  1.18  3       0.76
\end{verbatim}

Not quite sure what to do here - on the one hand we do not observe a
significant run x event interaction. On the other hand, the primary VOE
effect (unexp \textgreater{} expected) is indeed variable across runs,
from visual inspection, and stronger in runs 1 and 2than runs 3 and 4,
when considering psychology and physics events). So I actually think
that we have the grounds for focusing the analysis on runs 1-2.

Here we fit a model including main effects of domain and event
(domain-general and domain-specific regions) and also their interaction
(domain-specific regions only), and organize and print the outputs.

\begin{Shaded}
\begin{Highlighting}[]
\NormalTok{focal\_data\_100 }\OtherTok{\textless{}{-}}\NormalTok{ focal\_data }\SpecialCharTok{\%\textgreater{}\%}
  \FunctionTok{filter}\NormalTok{(event }\SpecialCharTok{!=} \StringTok{"fam"}\NormalTok{,}
\NormalTok{         top\_n\_voxels }\SpecialCharTok{==} \StringTok{"100"}\NormalTok{,}
\NormalTok{         domain }\SpecialCharTok{!=} \StringTok{"both"}\NormalTok{,}
\NormalTok{         (extracted\_run\_number }\SpecialCharTok{==} \StringTok{"run1"} \SpecialCharTok{|}\NormalTok{ extracted\_run\_number }\SpecialCharTok{==} \StringTok{"run2"}\NormalTok{))}


\NormalTok{ROIs }\OtherTok{\textless{}{-}} \FunctionTok{levels}\NormalTok{(}\FunctionTok{as.factor}\NormalTok{(focal\_data\_100}\SpecialCharTok{$}\NormalTok{ROI\_name\_final))}

\NormalTok{focal\_modelsummaries\_100 }\OtherTok{\textless{}{-}} \FunctionTok{data.frame}\NormalTok{()}

\ControlFlowTok{for}\NormalTok{ (ROI }\ControlFlowTok{in}\NormalTok{ ROIs) \{}
\NormalTok{  curROI }\OtherTok{\textless{}{-}}\NormalTok{ ROI}
\NormalTok{  ROIdata }\OtherTok{\textless{}{-}}\NormalTok{ focal\_data\_100 }\SpecialCharTok{\%\textgreater{}\%} \FunctionTok{filter}\NormalTok{(ROI\_name\_final }\SpecialCharTok{==}\NormalTok{ curROI)}
  \CommentTok{\# print("Current ROI: ", curROI)}
  \ControlFlowTok{if}\NormalTok{ (ROIdata}\SpecialCharTok{$}\NormalTok{ROI\_category[}\DecValTok{1}\NormalTok{] }\SpecialCharTok{\%in\%}\NormalTok{ domain\_specific\_regions) \{}
\NormalTok{    ROI\_category }\OtherTok{\textless{}{-}} \StringTok{"specific"}
\NormalTok{    model }\OtherTok{\textless{}{-}} \FunctionTok{lmer}\NormalTok{(}
      \AttributeTok{data =}\NormalTok{ ROIdata,}
      \AttributeTok{formula =}\NormalTok{ meanbeta }\SpecialCharTok{\textasciitilde{}}\NormalTok{ event }\SpecialCharTok{*}\NormalTok{ domain }\SpecialCharTok{+}\NormalTok{ (}\DecValTok{1} \SpecialCharTok{|}\NormalTok{ extracted\_run\_number) }\SpecialCharTok{+}\NormalTok{ (}\DecValTok{1}\SpecialCharTok{|}\NormalTok{subjectID)}
\NormalTok{    )}
    \CommentTok{\# model\_check \textless{}{-} \# check\_model(model)}
\NormalTok{  \} }\ControlFlowTok{else}\NormalTok{ \{}
\NormalTok{    ROI\_category }\OtherTok{\textless{}{-}} \StringTok{"general"}
\NormalTok{    model }\OtherTok{\textless{}{-}} \FunctionTok{lmer}\NormalTok{(}
      \AttributeTok{data =}\NormalTok{ ROIdata,}
      \AttributeTok{formula =}\NormalTok{ meanbeta }\SpecialCharTok{\textasciitilde{}}\NormalTok{ event }\SpecialCharTok{+}\NormalTok{ domain }\SpecialCharTok{+}\NormalTok{ (}\DecValTok{1} \SpecialCharTok{|}\NormalTok{ extracted\_run\_number) }\SpecialCharTok{+}\NormalTok{ (}\DecValTok{1}\SpecialCharTok{|}\NormalTok{subjectID)}
\NormalTok{    )}
\NormalTok{  \}}
\NormalTok{  modelsummary }\OtherTok{\textless{}{-}}
    \FunctionTok{cbind}\NormalTok{(}
\NormalTok{      ROI\_category,}
\NormalTok{      curROI,}
      \FunctionTok{rownames}\NormalTok{(}\FunctionTok{summary}\NormalTok{(model)}\SpecialCharTok{$}\NormalTok{coefficients),}
      \FunctionTok{summary}\NormalTok{(model)}\SpecialCharTok{$}\NormalTok{coefficients}
\NormalTok{    ) }\SpecialCharTok{\%\textgreater{}\%}
    \FunctionTok{as.data.frame}\NormalTok{()}
  \FunctionTok{colnames}\NormalTok{(modelsummary) }\OtherTok{\textless{}{-}}
    \FunctionTok{c}\NormalTok{(}\StringTok{"domain"}\NormalTok{, }\StringTok{"ROI"}\NormalTok{, }\StringTok{"effect"}\NormalTok{, }\StringTok{"B"}\NormalTok{, }\StringTok{"SE"}\NormalTok{, }\StringTok{"df"}\NormalTok{, }\StringTok{"t"}\NormalTok{, }\StringTok{"p"}\NormalTok{)}
\NormalTok{  focal\_modelsummaries\_100 }\OtherTok{\textless{}{-}}
    \FunctionTok{rbind}\NormalTok{(focal\_modelsummaries\_100, modelsummary)}
\NormalTok{\}}
\end{Highlighting}
\end{Shaded}

\begin{verbatim}
## boundary (singular) fit: see ?isSingular
## boundary (singular) fit: see ?isSingular
\end{verbatim}

\begin{Shaded}
\begin{Highlighting}[]
\NormalTok{focal\_modelsummaries\_100 }\OtherTok{\textless{}{-}}\NormalTok{ focal\_modelsummaries\_100 }\SpecialCharTok{\%\textgreater{}\%}
  \FunctionTok{mutate\_at}\NormalTok{(}\FunctionTok{c}\NormalTok{(}\DecValTok{4}\SpecialCharTok{:}\DecValTok{8}\NormalTok{), as.numeric) }\SpecialCharTok{\%\textgreater{}\%}
  \FunctionTok{mutate}\NormalTok{(}\AttributeTok{sig =} \FunctionTok{as.factor}\NormalTok{(}\FunctionTok{case\_when}\NormalTok{(p }\SpecialCharTok{\textless{}}\NormalTok{ .}\DecValTok{05} \SpecialCharTok{\textasciitilde{}} \StringTok{"yes"}\NormalTok{,}
\NormalTok{                                   p }\SpecialCharTok{\textgreater{}=}\NormalTok{ .}\DecValTok{05} \SpecialCharTok{\textasciitilde{}} \StringTok{"no"}\NormalTok{)),}
         \AttributeTok{star =} \FunctionTok{as.factor}\NormalTok{(}
           \FunctionTok{case\_when}\NormalTok{(p }\SpecialCharTok{\textless{}}\NormalTok{ .}\DecValTok{001} \SpecialCharTok{\textasciitilde{}} \StringTok{"***"}\NormalTok{,}
\NormalTok{                     p }\SpecialCharTok{\textless{}}\NormalTok{ .}\DecValTok{01} \SpecialCharTok{\textasciitilde{}} \StringTok{"**"}\NormalTok{,}
\NormalTok{                     p }\SpecialCharTok{\textless{}}\NormalTok{ .}\DecValTok{05} \SpecialCharTok{\textasciitilde{}} \StringTok{"*"}\NormalTok{,}
\NormalTok{                     p }\SpecialCharTok{\textless{}}\NormalTok{ .}\DecValTok{1} \SpecialCharTok{\textasciitilde{}} \StringTok{"\textasciitilde{}"}\NormalTok{,}
                     \ConstantTok{TRUE} \SpecialCharTok{\textasciitilde{}} \StringTok{" "}\NormalTok{)}
\NormalTok{         )) }\SpecialCharTok{\%\textgreater{}\%}
  \FunctionTok{mutate}\NormalTok{(}\AttributeTok{p =} \FunctionTok{round}\NormalTok{(p, }\AttributeTok{digits =} \DecValTok{3}\NormalTok{))}

\NormalTok{focal\_modelsummaries\_100}\SpecialCharTok{$}\NormalTok{effect }\OtherTok{\textless{}{-}}
  \FunctionTok{as.factor}\NormalTok{(focal\_modelsummaries\_100}\SpecialCharTok{$}\NormalTok{effect)}


\FunctionTok{levels}\NormalTok{(focal\_modelsummaries\_100}\SpecialCharTok{$}\NormalTok{effect) }\OtherTok{\textless{}{-}}
  \FunctionTok{c}\NormalTok{(}\StringTok{"intercept"}\NormalTok{, }\StringTok{"domain"}\NormalTok{, }\StringTok{"event"}\NormalTok{, }\StringTok{"event:domain"}\NormalTok{)}

\FunctionTok{rownames}\NormalTok{(focal\_modelsummaries\_100) }\OtherTok{\textless{}{-}} \ConstantTok{NULL}

\NormalTok{knitr}\SpecialCharTok{::}\FunctionTok{kable}\NormalTok{(dplyr}\SpecialCharTok{::}\FunctionTok{arrange}\NormalTok{(focal\_modelsummaries\_100, effect), }\AttributeTok{digits =} \DecValTok{3}\NormalTok{)}
\end{Highlighting}
\end{Shaded}

\begin{longtable}[]{@{}
  >{\raggedright\arraybackslash}p{(\columnwidth - 18\tabcolsep) * \real{0.0968}}
  >{\raggedright\arraybackslash}p{(\columnwidth - 18\tabcolsep) * \real{0.3118}}
  >{\raggedright\arraybackslash}p{(\columnwidth - 18\tabcolsep) * \real{0.1398}}
  >{\raggedleft\arraybackslash}p{(\columnwidth - 18\tabcolsep) * \real{0.0753}}
  >{\raggedleft\arraybackslash}p{(\columnwidth - 18\tabcolsep) * \real{0.0645}}
  >{\raggedleft\arraybackslash}p{(\columnwidth - 18\tabcolsep) * \real{0.0753}}
  >{\raggedleft\arraybackslash}p{(\columnwidth - 18\tabcolsep) * \real{0.0753}}
  >{\raggedleft\arraybackslash}p{(\columnwidth - 18\tabcolsep) * \real{0.0645}}
  >{\raggedright\arraybackslash}p{(\columnwidth - 18\tabcolsep) * \real{0.0430}}
  >{\raggedright\arraybackslash}p{(\columnwidth - 18\tabcolsep) * \real{0.0538}}@{}}
\toprule
\begin{minipage}[b]{\linewidth}\raggedright
domain
\end{minipage} & \begin{minipage}[b]{\linewidth}\raggedright
ROI
\end{minipage} & \begin{minipage}[b]{\linewidth}\raggedright
effect
\end{minipage} & \begin{minipage}[b]{\linewidth}\raggedleft
B
\end{minipage} & \begin{minipage}[b]{\linewidth}\raggedleft
SE
\end{minipage} & \begin{minipage}[b]{\linewidth}\raggedleft
df
\end{minipage} & \begin{minipage}[b]{\linewidth}\raggedleft
t
\end{minipage} & \begin{minipage}[b]{\linewidth}\raggedleft
p
\end{minipage} & \begin{minipage}[b]{\linewidth}\raggedright
sig
\end{minipage} & \begin{minipage}[b]{\linewidth}\raggedright
star
\end{minipage} \\
\midrule
\endhead
general & insula\_bilateral & intercept & 0.728 & 0.174 & 31.00 & 4.176
& 0.000 & yes & *** \\
general & MT\_bilateral & intercept & 5.222 & 0.470 & 25.57 & 11.107 &
0.000 & yes & *** \\
general & precentral\_B\_IFGop\_bilateral & intercept & 2.801 & 0.282 &
11.04 & 9.939 & 0.000 & yes & *** \\
specific & superiortemporal\_L & intercept & 0.616 & 0.466 & 6.46 &
1.321 & 0.231 & no & \\
specific & superiortemporal\_R & intercept & 2.024 & 0.319 & 17.91 &
6.349 & 0.000 & yes & *** \\
specific & supramarginal\_L & intercept & 2.384 & 0.362 & 31.00 & 6.589
& 0.000 & yes & *** \\
specific & supramarginal\_R & intercept & 2.576 & 0.509 & 14.42 & 5.063
& 0.000 & yes & *** \\
general & V1\_bilateral & intercept & 9.429 & 0.668 & 24.70 & 14.107 &
0.000 & yes & *** \\
general & insula\_bilateral & domain & 0.007 & 0.052 & 222.00 & 0.136 &
0.892 & no & \\
general & MT\_bilateral & domain & 0.727 & 0.067 & 221.00 & 10.809 &
0.000 & yes & *** \\
general & precentral\_B\_IFGop\_bilateral & domain & 0.418 & 0.085 &
221.00 & 4.894 & 0.000 & yes & *** \\
specific & superiortemporal\_L & domain & -0.292 & 0.097 & 220.00 &
-3.027 & 0.003 & yes & ** \\
specific & superiortemporal\_R & domain & -0.247 & 0.082 & 220.00 &
-3.004 & 0.003 & yes & ** \\
specific & supramarginal\_L & domain & 0.262 & 0.078 & 221.00 & 3.343 &
0.001 & yes & *** \\
specific & supramarginal\_R & domain & 0.511 & 0.084 & 220.00 & 6.067 &
0.000 & yes & *** \\
general & V1\_bilateral & domain & 0.580 & 0.134 & 221.00 & 4.315 &
0.000 & yes & *** \\
general & insula\_bilateral & event & -0.059 & 0.052 & 222.00 & -1.128 &
0.260 & no & \\
general & MT\_bilateral & event & -0.052 & 0.067 & 221.00 & -0.772 &
0.441 & no & \\
general & precentral\_B\_IFGop\_bilateral & event & -0.215 & 0.085 &
221.00 & -2.519 & 0.012 & yes & * \\
specific & superiortemporal\_L & event & -0.062 & 0.097 & 220.00 &
-0.637 & 0.525 & no & \\
specific & superiortemporal\_R & event & -0.125 & 0.082 & 220.00 &
-1.512 & 0.132 & no & \\
specific & supramarginal\_L & event & -0.067 & 0.078 & 221.00 & -0.852 &
0.395 & no & \\
specific & supramarginal\_R & event & -0.166 & 0.084 & 220.00 & -1.968 &
0.050 & no & \textasciitilde{} \\
general & V1\_bilateral & event & -0.084 & 0.134 & 221.00 & -0.627 &
0.531 & no & \\
specific & superiortemporal\_L & event:domain & 0.065 & 0.097 & 220.00 &
0.674 & 0.501 & no & \\
specific & superiortemporal\_R & event:domain & -0.049 & 0.082 & 220.00
& -0.594 & 0.553 & no & \\
specific & supramarginal\_L & event:domain & -0.189 & 0.078 & 221.00 &
-2.407 & 0.017 & yes & * \\
specific & supramarginal\_R & event:domain & -0.051 & 0.084 & 220.00 &
-0.601 & 0.549 & no & \\
\bottomrule
\end{longtable}

Below we check, for each domain-specific region (left and right STS and
SMG), whether a model including an interaction between domain and event
fits better than a model without the interaction.

\begin{Shaded}
\begin{Highlighting}[]
\NormalTok{LSMG\_model }\OtherTok{\textless{}{-}} \FunctionTok{lmer}\NormalTok{(}
      \AttributeTok{data =}\NormalTok{ focal\_data\_100 }\SpecialCharTok{\%\textgreater{}\%} \FunctionTok{filter}\NormalTok{(ROI\_name\_final }\SpecialCharTok{==} \StringTok{"supramarginal\_L"}\NormalTok{),}
      \AttributeTok{formula =}\NormalTok{ meanbeta }\SpecialCharTok{\textasciitilde{}}\NormalTok{ event }\SpecialCharTok{*}\NormalTok{ domain }\SpecialCharTok{+}\NormalTok{ (}\DecValTok{1}\SpecialCharTok{|}\NormalTok{extracted\_run\_number) }\SpecialCharTok{+}\NormalTok{ (}\DecValTok{1}\SpecialCharTok{|}\NormalTok{subjectID)}
\NormalTok{    )}
\end{Highlighting}
\end{Shaded}

\begin{verbatim}
## boundary (singular) fit: see ?isSingular
\end{verbatim}

\begin{Shaded}
\begin{Highlighting}[]
\CommentTok{\# check\_model(LSMG\_model)}
\FunctionTok{check\_outliers}\NormalTok{(LSMG\_model)}
\end{Highlighting}
\end{Shaded}

\begin{verbatim}
## OK: No outliers detected.
\end{verbatim}

\begin{Shaded}
\begin{Highlighting}[]
\FunctionTok{lsmeans}\NormalTok{(LSMG\_model, pairwise }\SpecialCharTok{\textasciitilde{}}\NormalTok{ event }\SpecialCharTok{|}\NormalTok{ domain)}\SpecialCharTok{$}\NormalTok{contrasts}
\end{Highlighting}
\end{Shaded}

\begin{verbatim}
## domain = physics:
##  contrast    estimate    SE  df t.ratio p.value
##  exp - unexp   -0.512 0.222 220 -2.305  0.0221 
## 
## domain = psychology:
##  contrast    estimate    SE  df t.ratio p.value
##  exp - unexp    0.244 0.222 220  1.100  0.2726 
## 
## Degrees-of-freedom method: kenward-roger
\end{verbatim}

\begin{Shaded}
\begin{Highlighting}[]
\NormalTok{LSMG\_model\_simpler }\OtherTok{\textless{}{-}} \FunctionTok{lmer}\NormalTok{(}
      \AttributeTok{data =}\NormalTok{ focal\_data\_100 }\SpecialCharTok{\%\textgreater{}\%} \FunctionTok{filter}\NormalTok{(ROI\_name\_final }\SpecialCharTok{==} \StringTok{"supramarginal\_L"}\NormalTok{),}
      \AttributeTok{formula =}\NormalTok{ meanbeta }\SpecialCharTok{\textasciitilde{}}\NormalTok{ event }\SpecialCharTok{+}\NormalTok{ domain }\SpecialCharTok{+}\NormalTok{ (}\DecValTok{1}\SpecialCharTok{|}\NormalTok{extracted\_run\_number) }\SpecialCharTok{+}\NormalTok{ (}\DecValTok{1}\SpecialCharTok{|}\NormalTok{subjectID)}
\NormalTok{    )}
\end{Highlighting}
\end{Shaded}

\begin{verbatim}
## boundary (singular) fit: see ?isSingular
\end{verbatim}

\begin{Shaded}
\begin{Highlighting}[]
\CommentTok{\# check\_model(LSMG\_model\_simpler)}
\FunctionTok{check\_outliers}\NormalTok{(LSMG\_model\_simpler)}
\end{Highlighting}
\end{Shaded}

\begin{verbatim}
## OK: No outliers detected.
\end{verbatim}

\begin{Shaded}
\begin{Highlighting}[]
\FunctionTok{anova}\NormalTok{(LSMG\_model\_simpler, LSMG\_model)}
\end{Highlighting}
\end{Shaded}

\begin{verbatim}
## refitting model(s) with ML (instead of REML)
\end{verbatim}

\begin{verbatim}
## Data: focal_data_100 %>% filter(ROI_name_final == "supramarginal_L")
## Models:
## LSMG_model_simpler: meanbeta ~ event + domain + (1 | extracted_run_number) + (1 | 
## LSMG_model_simpler:     subjectID)
## LSMG_model: meanbeta ~ event * domain + (1 | extracted_run_number) + (1 | 
## LSMG_model:     subjectID)
##                    npar AIC BIC logLik deviance Chisq Df Pr(>Chisq)  
## LSMG_model_simpler    6 955 976   -471      943                      
## LSMG_model            7 951 976   -468      937   5.8  1      0.016 *
## ---
## Signif. codes:  0 '***' 0.001 '**' 0.01 '*' 0.05 '.' 0.1 ' ' 1
\end{verbatim}

\begin{Shaded}
\begin{Highlighting}[]
\NormalTok{RSMG\_model }\OtherTok{\textless{}{-}} \FunctionTok{lmer}\NormalTok{(}
      \AttributeTok{data =}\NormalTok{ focal\_data\_100 }\SpecialCharTok{\%\textgreater{}\%} \FunctionTok{filter}\NormalTok{(ROI\_name\_final }\SpecialCharTok{==} \StringTok{"supramarginal\_R"}\NormalTok{),}
      \AttributeTok{formula =}\NormalTok{ meanbeta }\SpecialCharTok{\textasciitilde{}}\NormalTok{ event }\SpecialCharTok{*}\NormalTok{ domain }\SpecialCharTok{+}\NormalTok{ (}\DecValTok{1}\SpecialCharTok{|}\NormalTok{extracted\_run\_number) }\SpecialCharTok{+}\NormalTok{ (}\DecValTok{1}\SpecialCharTok{|}\NormalTok{subjectID)}
\NormalTok{    )}
\CommentTok{\# summary(RSMG\_model)}

\NormalTok{RSMG\_model }\SpecialCharTok{\%\textgreater{}\%}
  \FunctionTok{apa\_print}\NormalTok{() }\SpecialCharTok{\%\textgreater{}\%}
  \FunctionTok{apa\_table}\NormalTok{()}
\end{Highlighting}
\end{Shaded}

\begin{table}[tbp]

\begin{center}
\begin{threeparttable}

\caption{\label{tab:unnamed-chunk-12}}

\begin{tabular}{llllll}
\toprule
Term & \multicolumn{1}{c}{$\hat{\beta}$} & \multicolumn{1}{c}{95\% CI} & \multicolumn{1}{c}{$t$} & \multicolumn{1}{c}{$\mathit{df}$} & \multicolumn{1}{c}{$p$}\\
\midrule
Intercept & 2.58 & {}[1.58, 3.57] & 5.06 & 14.42 & < .001\\
Event1 & -0.17 & {}[-0.33, 0.00] & -1.97 & 220.00 & .050\\
Domain1 & 0.51 & {}[0.35, 0.68] & 6.07 & 220.00 & < .001\\
Event1 $\times$ Domain1 & -0.05 & {}[-0.22, 0.11] & -0.60 & 220.00 & .549\\
\bottomrule
\end{tabular}

\end{threeparttable}
\end{center}

\end{table}

\begin{Shaded}
\begin{Highlighting}[]
\CommentTok{\# check\_model(RSMG\_model)}
\FunctionTok{check\_outliers}\NormalTok{(RSMG\_model)}
\end{Highlighting}
\end{Shaded}

\begin{verbatim}
## OK: No outliers detected.
\end{verbatim}

\begin{Shaded}
\begin{Highlighting}[]
\FunctionTok{lsmeans}\NormalTok{(RSMG\_model, pairwise }\SpecialCharTok{\textasciitilde{}}\NormalTok{ event }\SpecialCharTok{|}\NormalTok{ domain)}\SpecialCharTok{$}\NormalTok{contrasts}
\end{Highlighting}
\end{Shaded}

\begin{verbatim}
## domain = physics:
##  contrast    estimate    SE  df t.ratio p.value
##  exp - unexp   -0.433 0.238 220 -1.816  0.0710 
## 
## domain = psychology:
##  contrast    estimate    SE  df t.ratio p.value
##  exp - unexp   -0.230 0.238 220 -0.967  0.3350 
## 
## Degrees-of-freedom method: kenward-roger
\end{verbatim}

\begin{Shaded}
\begin{Highlighting}[]
\NormalTok{RSMG\_model\_simpler }\OtherTok{\textless{}{-}} \FunctionTok{lmer}\NormalTok{(}
      \AttributeTok{data =}\NormalTok{ focal\_data\_100 }\SpecialCharTok{\%\textgreater{}\%} \FunctionTok{filter}\NormalTok{(ROI\_name\_final }\SpecialCharTok{==} \StringTok{"supramarginal\_R"}\NormalTok{),}
      \AttributeTok{formula =}\NormalTok{ meanbeta }\SpecialCharTok{\textasciitilde{}}\NormalTok{ event }\SpecialCharTok{+}\NormalTok{ domain }\SpecialCharTok{+}\NormalTok{ (}\DecValTok{1}\SpecialCharTok{|}\NormalTok{extracted\_run\_number) }\SpecialCharTok{+}\NormalTok{ (}\DecValTok{1}\SpecialCharTok{|}\NormalTok{subjectID)}
\NormalTok{    )}
\FunctionTok{anova}\NormalTok{(RSMG\_model\_simpler, RSMG\_model)}
\end{Highlighting}
\end{Shaded}

\begin{verbatim}
## refitting model(s) with ML (instead of REML)
\end{verbatim}

\begin{verbatim}
## Data: focal_data_100 %>% filter(ROI_name_final == "supramarginal_R")
## Models:
## RSMG_model_simpler: meanbeta ~ event + domain + (1 | extracted_run_number) + (1 | 
## RSMG_model_simpler:     subjectID)
## RSMG_model: meanbeta ~ event * domain + (1 | extracted_run_number) + (1 | 
## RSMG_model:     subjectID)
##                    npar  AIC  BIC logLik deviance Chisq Df Pr(>Chisq)
## RSMG_model_simpler    6  998 1019   -493      986                    
## RSMG_model            7 1000 1025   -493      986  0.37  1       0.55
\end{verbatim}

\begin{Shaded}
\begin{Highlighting}[]
\NormalTok{RSTS\_model }\OtherTok{\textless{}{-}} \FunctionTok{lmer}\NormalTok{(}
      \AttributeTok{data =}\NormalTok{ focal\_data\_100 }\SpecialCharTok{\%\textgreater{}\%} \FunctionTok{filter}\NormalTok{(ROI\_name\_final }\SpecialCharTok{==} \StringTok{"superiortemporal\_R"}\NormalTok{),}
      \AttributeTok{formula =}\NormalTok{ meanbeta }\SpecialCharTok{\textasciitilde{}}\NormalTok{ event }\SpecialCharTok{*}\NormalTok{ domain }\SpecialCharTok{+}\NormalTok{ (}\DecValTok{1}\SpecialCharTok{|}\NormalTok{extracted\_run\_number) }\SpecialCharTok{+}\NormalTok{ (}\DecValTok{1}\SpecialCharTok{|}\NormalTok{subjectID)}
\NormalTok{    )}
\CommentTok{\# summary(RSTS\_model)}

\NormalTok{RSTS\_model }\SpecialCharTok{\%\textgreater{}\%}
  \FunctionTok{apa\_print}\NormalTok{() }\SpecialCharTok{\%\textgreater{}\%}
  \FunctionTok{apa\_table}\NormalTok{()}
\end{Highlighting}
\end{Shaded}

\begin{table}[tbp]

\begin{center}
\begin{threeparttable}

\caption{\label{tab:unnamed-chunk-13}}

\begin{tabular}{llllll}
\toprule
Term & \multicolumn{1}{c}{$\hat{\beta}$} & \multicolumn{1}{c}{95\% CI} & \multicolumn{1}{c}{$t$} & \multicolumn{1}{c}{$\mathit{df}$} & \multicolumn{1}{c}{$p$}\\
\midrule
Intercept & 2.02 & {}[1.40, 2.65] & 6.35 & 17.91 & < .001\\
Event1 & -0.12 & {}[-0.29, 0.04] & -1.51 & 220.00 & .132\\
Domain1 & -0.25 & {}[-0.41, -0.09] & -3.00 & 220.00 & .003\\
Event1 $\times$ Domain1 & -0.05 & {}[-0.21, 0.11] & -0.59 & 220.00 & .553\\
\bottomrule
\end{tabular}

\end{threeparttable}
\end{center}

\end{table}

\begin{Shaded}
\begin{Highlighting}[]
\CommentTok{\# check\_model(RSTS\_model)}
\FunctionTok{check\_outliers}\NormalTok{(RSTS\_model)}
\end{Highlighting}
\end{Shaded}

\begin{verbatim}
## OK: No outliers detected.
\end{verbatim}

\begin{Shaded}
\begin{Highlighting}[]
\FunctionTok{lsmeans}\NormalTok{(RSTS\_model, pairwise }\SpecialCharTok{\textasciitilde{}}\NormalTok{ event }\SpecialCharTok{|}\NormalTok{ domain)}\SpecialCharTok{$}\NormalTok{contrasts}
\end{Highlighting}
\end{Shaded}

\begin{verbatim}
## domain = physics:
##  contrast    estimate    SE  df t.ratio p.value
##  exp - unexp   -0.347 0.233 220 -1.489  0.1380 
## 
## domain = psychology:
##  contrast    estimate    SE  df t.ratio p.value
##  exp - unexp   -0.151 0.233 220 -0.649  0.5170 
## 
## Degrees-of-freedom method: kenward-roger
\end{verbatim}

\begin{Shaded}
\begin{Highlighting}[]
\NormalTok{RSTS\_model\_simpler }\OtherTok{\textless{}{-}} \FunctionTok{lmer}\NormalTok{(}
      \AttributeTok{data =}\NormalTok{ focal\_data\_100 }\SpecialCharTok{\%\textgreater{}\%} \FunctionTok{filter}\NormalTok{(ROI\_name\_final }\SpecialCharTok{==} \StringTok{"superiortemporal\_R"}\NormalTok{),}
      \AttributeTok{formula =}\NormalTok{ meanbeta }\SpecialCharTok{\textasciitilde{}}\NormalTok{ event }\SpecialCharTok{+}\NormalTok{ domain }\SpecialCharTok{+}\NormalTok{ (}\DecValTok{1}\SpecialCharTok{|}\NormalTok{extracted\_run\_number) }\SpecialCharTok{+}\NormalTok{ (}\DecValTok{1}\SpecialCharTok{|}\NormalTok{subjectID)}
\NormalTok{    )}
\FunctionTok{anova}\NormalTok{(RSTS\_model\_simpler, RSTS\_model)}
\end{Highlighting}
\end{Shaded}

\begin{verbatim}
## refitting model(s) with ML (instead of REML)
\end{verbatim}

\begin{verbatim}
## Data: focal_data_100 %>% filter(ROI_name_final == "superiortemporal_R")
## Models:
## RSTS_model_simpler: meanbeta ~ event + domain + (1 | extracted_run_number) + (1 | 
## RSTS_model_simpler:     subjectID)
## RSTS_model: meanbeta ~ event * domain + (1 | extracted_run_number) + (1 | 
## RSTS_model:     subjectID)
##                    npar AIC BIC logLik deviance Chisq Df Pr(>Chisq)
## RSTS_model_simpler    6 960 981   -474      948                    
## RSTS_model            7 962 987   -474      948  0.36  1       0.55
\end{verbatim}

\begin{Shaded}
\begin{Highlighting}[]
\NormalTok{LSTS\_model }\OtherTok{\textless{}{-}} \FunctionTok{lmer}\NormalTok{(}
      \AttributeTok{data =}\NormalTok{ focal\_data\_100 }\SpecialCharTok{\%\textgreater{}\%} \FunctionTok{filter}\NormalTok{(ROI\_name\_final }\SpecialCharTok{==} \StringTok{"superiortemporal\_L"}\NormalTok{),}
      \AttributeTok{formula =}\NormalTok{ meanbeta }\SpecialCharTok{\textasciitilde{}}\NormalTok{ event }\SpecialCharTok{*}\NormalTok{ domain }\SpecialCharTok{+}\NormalTok{ (}\DecValTok{1}\SpecialCharTok{|}\NormalTok{extracted\_run\_number) }\SpecialCharTok{+}\NormalTok{ (}\DecValTok{1}\SpecialCharTok{|}\NormalTok{subjectID)}
\NormalTok{    )}
\CommentTok{\# summary(LSTS\_model)}
\NormalTok{LSTS\_model }\SpecialCharTok{\%\textgreater{}\%}
  \FunctionTok{apa\_print}\NormalTok{() }\SpecialCharTok{\%\textgreater{}\%}
  \FunctionTok{apa\_table}\NormalTok{()}
\end{Highlighting}
\end{Shaded}

\begin{table}[tbp]

\begin{center}
\begin{threeparttable}

\caption{\label{tab:unnamed-chunk-14}}

\begin{tabular}{llllll}
\toprule
Term & \multicolumn{1}{c}{$\hat{\beta}$} & \multicolumn{1}{c}{95\% CI} & \multicolumn{1}{c}{$t$} & \multicolumn{1}{c}{$\mathit{df}$} & \multicolumn{1}{c}{$p$}\\
\midrule
Intercept & 0.62 & {}[-0.30, 1.53] & 1.32 & 6.46 & .231\\
Event1 & -0.06 & {}[-0.25, 0.13] & -0.64 & 220.00 & .525\\
Domain1 & -0.29 & {}[-0.48, -0.10] & -3.03 & 220.00 & .003\\
Event1 $\times$ Domain1 & 0.07 & {}[-0.12, 0.25] & 0.67 & 220.00 & .501\\
\bottomrule
\end{tabular}

\end{threeparttable}
\end{center}

\end{table}

\begin{Shaded}
\begin{Highlighting}[]
\CommentTok{\# check\_model(LSTS\_model)}
\FunctionTok{check\_outliers}\NormalTok{(LSTS\_model)}
\end{Highlighting}
\end{Shaded}

\begin{verbatim}
## Warning: 1 outliers detected (cases 180).
\end{verbatim}

\begin{Shaded}
\begin{Highlighting}[]
\NormalTok{outliers\_list }\OtherTok{\textless{}{-}} \FunctionTok{check\_outliers}\NormalTok{(LSTS\_model)}
\NormalTok{insight}\SpecialCharTok{::}\FunctionTok{get\_data}\NormalTok{(LSTS\_model)[outliers\_list, ]}
\end{Highlighting}
\end{Shaded}

\begin{verbatim}
##     meanbeta event     domain extracted_run_number       subjectID
## 180    -11.6   exp psychology                 run2 sub-SAXNES2s013
\end{verbatim}

\begin{Shaded}
\begin{Highlighting}[]
\NormalTok{LSTS\_model\_nooutliers }\OtherTok{\textless{}{-}} \FunctionTok{lmer}\NormalTok{(}
      \AttributeTok{data =}\NormalTok{ insight}\SpecialCharTok{::}\FunctionTok{get\_data}\NormalTok{(LSTS\_model) }\SpecialCharTok{\%\textgreater{}\%}
        \FunctionTok{filter}\NormalTok{(meanbeta }\SpecialCharTok{!=} \SpecialCharTok{{-}}\FloatTok{11.62629}\NormalTok{),}
      \AttributeTok{formula =}\NormalTok{ meanbeta }\SpecialCharTok{\textasciitilde{}}\NormalTok{ event }\SpecialCharTok{*}\NormalTok{ domain }\SpecialCharTok{+}\NormalTok{ (}\DecValTok{1}\SpecialCharTok{|}\NormalTok{extracted\_run\_number) }\SpecialCharTok{+}\NormalTok{ (}\DecValTok{1}\SpecialCharTok{|}\NormalTok{subjectID)}
\NormalTok{    )}
\CommentTok{\# summary(LSTS\_model\_nooutliers)}
\NormalTok{LSTS\_model\_nooutliers }\SpecialCharTok{\%\textgreater{}\%}
  \FunctionTok{apa\_print}\NormalTok{() }\SpecialCharTok{\%\textgreater{}\%}
  \FunctionTok{apa\_table}\NormalTok{()}
\end{Highlighting}
\end{Shaded}

\begin{table}[tbp]

\begin{center}
\begin{threeparttable}

\caption{\label{tab:unnamed-chunk-14}}

\begin{tabular}{llllll}
\toprule
Term & \multicolumn{1}{c}{$\hat{\beta}$} & \multicolumn{1}{c}{95\% CI} & \multicolumn{1}{c}{$t$} & \multicolumn{1}{c}{$\mathit{df}$} & \multicolumn{1}{c}{$p$}\\
\midrule
Intercept & 0.62 & {}[-0.30, 1.53] & 1.32 & 6.46 & .231\\
Event1 & -0.06 & {}[-0.25, 0.13] & -0.64 & 220.00 & .525\\
Domain1 & -0.29 & {}[-0.48, -0.10] & -3.03 & 220.00 & .003\\
Event1 $\times$ Domain1 & 0.07 & {}[-0.12, 0.25] & 0.67 & 220.00 & .501\\
\bottomrule
\end{tabular}

\end{threeparttable}
\end{center}

\end{table}

\begin{Shaded}
\begin{Highlighting}[]
\FunctionTok{lsmeans}\NormalTok{(LSTS\_model\_nooutliers, pairwise }\SpecialCharTok{\textasciitilde{}}\NormalTok{ event }\SpecialCharTok{|}\NormalTok{ domain)}\SpecialCharTok{$}\NormalTok{contrasts}
\end{Highlighting}
\end{Shaded}

\begin{verbatim}
## domain = physics:
##  contrast    estimate    SE  df t.ratio p.value
##  exp - unexp   0.0071 0.273 220  0.026  0.9790 
## 
## domain = psychology:
##  contrast    estimate    SE  df t.ratio p.value
##  exp - unexp  -0.2532 0.273 220 -0.927  0.3550 
## 
## Degrees-of-freedom method: kenward-roger
\end{verbatim}

\begin{Shaded}
\begin{Highlighting}[]
\NormalTok{LSTS\_model\_simpler }\OtherTok{\textless{}{-}} \FunctionTok{lmer}\NormalTok{(}
      \AttributeTok{data =}\NormalTok{ insight}\SpecialCharTok{::}\FunctionTok{get\_data}\NormalTok{(LSTS\_model) }\SpecialCharTok{\%\textgreater{}\%}
        \FunctionTok{filter}\NormalTok{(meanbeta }\SpecialCharTok{!=} \SpecialCharTok{{-}}\FloatTok{11.62629}\NormalTok{),}
      \AttributeTok{formula =}\NormalTok{ meanbeta }\SpecialCharTok{\textasciitilde{}}\NormalTok{ event }\SpecialCharTok{+}\NormalTok{ domain }\SpecialCharTok{+}\NormalTok{ (}\DecValTok{1}\SpecialCharTok{|}\NormalTok{extracted\_run\_number) }\SpecialCharTok{+}\NormalTok{ (}\DecValTok{1}\SpecialCharTok{|}\NormalTok{subjectID)}
\NormalTok{    )}
\FunctionTok{anova}\NormalTok{(LSTS\_model\_simpler, LSTS\_model\_nooutliers)}
\end{Highlighting}
\end{Shaded}

\begin{verbatim}
## refitting model(s) with ML (instead of REML)
\end{verbatim}

\begin{verbatim}
## Data: insight::get_data(LSTS_model) %>% filter(meanbeta != -11.62629)
## Models:
## LSTS_model_simpler: meanbeta ~ event + domain + (1 | extracted_run_number) + (1 | 
## LSTS_model_simpler:     subjectID)
## LSTS_model_nooutliers: meanbeta ~ event * domain + (1 | extracted_run_number) + (1 | 
## LSTS_model_nooutliers:     subjectID)
##                       npar  AIC  BIC logLik deviance Chisq Df Pr(>Chisq)
## LSTS_model_simpler       6 1048 1069   -518     1036                    
## LSTS_model_nooutliers    7 1049 1074   -518     1035  0.46  1        0.5
\end{verbatim}

Here are the main findings from the confirmatory analysis on focal
regions:

\hypertarget{domain-effects}{%
\subsection{Domain effects}\label{domain-effects}}

\begin{itemize}
\tightlist
\item
  With the exception of bilateral insula, all regions showed a domain
  univariate effect.
\item
  Left and right STS (\texttt{superiortemporal}) responded more, on
  average, to psychological than physical events, as expected. Left and
  right SMG (\texttt{supramarginal}) responded more, on average, to
  physical than psychological events, as expected. These regions were
  chosen in each person as the set of voxels that responded more to
  social vs phyiscal interaction in an independent task.
\item
  But we also found evidence for domain effects in MD regions, that
  responded more, in our independent localizer, when each person
  performed difficult vs easy spatial working memory tasks. Bilateral
  IFG (\texttt{precentral\_B\_IFGop\_bilateral}) responded more to
  physical than psychological events. This finding coheres with prior
  work on the intuitive physics regions.
\item
  We also found evidence for domain effects in visual regions, which
  were selected because they were the most visually responsive (V1,
  \texttt{V1\_bilateral}) or responsive to coherent motion (MT,
  \texttt{MT\_bilateral}). We saw hints of these effects in the last
  experiment from our visual regions.
\end{itemize}

\hypertarget{event-effects}{%
\subsection{Event effects}\label{event-effects}}

\begin{itemize}
\tightlist
\item
  The following regions showed a main effect of event, responding more
  on average to unexpected than expected test events regardless of
  domain: bilateral IFG (\texttt{precentral\_B\_IFGop\_bilateral}).
  Right SMG (\texttt{supramarginal\_R}) also showed a marginal effect in
  the same direction (p = .05).
\item
  Notably we did not see any effects in MT or V1
\end{itemize}

\hypertarget{domain-specific-prediction-error}{%
\subsection{Domain-specific prediction
error}\label{domain-specific-prediction-error}}

\begin{itemize}
\tightlist
\item
  The following regions showed an event x domain interaction: left SMG
  (\texttt{supramarginal\_L}). This interaction was exactly as
  predicted: greater responses to unexpected than expected physical
  events, but not psychological events. This model, including the
  interaction, fits the data better by a likelihood ratio test than a
  model including just the main effects.
\end{itemize}

\hypertarget{part-2-all-regions-exploratory}{%
\section{PART 2: All regions
(exploratory)}\label{part-2-all-regions-exploratory}}

\begin{Shaded}
\begin{Highlighting}[]
\NormalTok{univariate\_data\_nonfocal }\SpecialCharTok{\%\textgreater{}\%}
  \FunctionTok{group\_by}\NormalTok{(subjectID, ROI\_name\_final, extracted\_run\_number, domain) }\SpecialCharTok{\%\textgreater{}\%}
  \FunctionTok{summarise}\NormalTok{(}\AttributeTok{n =} \FunctionTok{n}\NormalTok{())}
\end{Highlighting}
\end{Shaded}

\hypertarget{plots-1}{%
\subsection{Plots}\label{plots-1}}

Here we plot the responses in just a few of these regions.

\begin{Shaded}
\begin{Highlighting}[]
\NormalTok{plot\_univar\_eachrun\_nonfocal }\OtherTok{\textless{}{-}} \ControlFlowTok{function}\NormalTok{(region) \{}
\NormalTok{  plotobject }\OtherTok{\textless{}{-}}
    \FunctionTok{ggplot}\NormalTok{(}\AttributeTok{data =}\NormalTok{ univariate\_summary\_domain }\SpecialCharTok{\%\textgreater{}\%}
             \FunctionTok{filter}\NormalTok{(}\FunctionTok{str\_detect}\NormalTok{(ROI\_name\_final, region),}
                    \FunctionTok{str\_length}\NormalTok{(extracted\_run\_number) }\SpecialCharTok{==} \DecValTok{4}\NormalTok{,}
\NormalTok{                    domain }\SpecialCharTok{!=} \StringTok{"both"}\NormalTok{), }
           \FunctionTok{aes}\NormalTok{(}\AttributeTok{x =}\NormalTok{ event, }\AttributeTok{y =}\NormalTok{ meanbeta, }\AttributeTok{fill =}\NormalTok{ domain)) }\SpecialCharTok{+}
    \FunctionTok{geom\_bar}\NormalTok{(}\AttributeTok{stat =} \StringTok{"identity"}\NormalTok{, }\FunctionTok{aes}\NormalTok{(}\AttributeTok{alpha =}\NormalTok{ event), }\AttributeTok{colour =} \StringTok{"black"}\NormalTok{) }\SpecialCharTok{+}
    \FunctionTok{geom\_errorbar}\NormalTok{(}
      \FunctionTok{aes}\NormalTok{(}\AttributeTok{ymin =}\NormalTok{ meanbeta }\SpecialCharTok{{-}}\NormalTok{ se, }\AttributeTok{ymax =}\NormalTok{ meanbeta }\SpecialCharTok{+}\NormalTok{ se),}
      \AttributeTok{position =} \FunctionTok{position\_dodge}\NormalTok{(}\AttributeTok{width =}\NormalTok{ .}\DecValTok{9}\NormalTok{),}
      \AttributeTok{width =}\NormalTok{ .}\DecValTok{2}\NormalTok{,}
      \AttributeTok{colour =} \StringTok{"black"}
\NormalTok{    ) }\SpecialCharTok{+}
    \FunctionTok{geom\_point}\NormalTok{(}\AttributeTok{data =}\NormalTok{ univariate\_data }\SpecialCharTok{\%\textgreater{}\%} 
             \FunctionTok{filter}\NormalTok{(}\FunctionTok{str\_detect}\NormalTok{(ROI\_name\_final, region),}
                    \FunctionTok{str\_length}\NormalTok{(extracted\_run\_number) }\SpecialCharTok{==} \DecValTok{4}\NormalTok{,}
\NormalTok{                    domain }\SpecialCharTok{!=} \StringTok{"both"}\NormalTok{), }
             \AttributeTok{alpha =}\NormalTok{ .}\DecValTok{1}\NormalTok{) }\SpecialCharTok{+}
    \FunctionTok{geom\_line}\NormalTok{(}\AttributeTok{data =}\NormalTok{ univariate\_data }\SpecialCharTok{\%\textgreater{}\%} 
             \FunctionTok{filter}\NormalTok{(}\FunctionTok{str\_detect}\NormalTok{(ROI\_name\_final, region),}
                    \FunctionTok{str\_length}\NormalTok{(extracted\_run\_number) }\SpecialCharTok{==} \DecValTok{4}\NormalTok{,}
\NormalTok{                    domain }\SpecialCharTok{!=} \StringTok{"both"}\NormalTok{), }
             \FunctionTok{aes}\NormalTok{(}\AttributeTok{group =}
\NormalTok{                    subjectID),}
              \AttributeTok{alpha =}\NormalTok{ .}\DecValTok{1}\NormalTok{) }\SpecialCharTok{+}
    \FunctionTok{theme\_cowplot}\NormalTok{(}\DecValTok{10}\NormalTok{) }\SpecialCharTok{+}
    \FunctionTok{facet\_wrap}\NormalTok{(}\SpecialCharTok{\textasciitilde{}}\NormalTok{ ROI\_name\_final }\SpecialCharTok{+}\NormalTok{ extracted\_run\_number }\SpecialCharTok{+}\NormalTok{ domain, }\AttributeTok{nrow =} \DecValTok{1}\NormalTok{) }\SpecialCharTok{+}
    \FunctionTok{scale\_fill\_manual}\NormalTok{(}\AttributeTok{values =} \FunctionTok{c}\NormalTok{(}\StringTok{"\#00AFBB"}\NormalTok{, }\StringTok{"\#FC4E07"}\NormalTok{)) }\SpecialCharTok{+}
    \FunctionTok{ylab}\NormalTok{(}\StringTok{"Average beta"}\NormalTok{) }\SpecialCharTok{+}
    \FunctionTok{xlab}\NormalTok{(}\StringTok{"Event"}\NormalTok{) }\SpecialCharTok{+}
    \FunctionTok{theme}\NormalTok{(}\AttributeTok{axis.text.x =} \FunctionTok{element\_text}\NormalTok{(}
      \AttributeTok{angle =} \DecValTok{90}\NormalTok{,}
      \AttributeTok{vjust =} \FloatTok{0.5}\NormalTok{,}
      \AttributeTok{hjust =} \DecValTok{1}
\NormalTok{    )) }\SpecialCharTok{+}
    \FunctionTok{coord\_cartesian}\NormalTok{(}\AttributeTok{ylim =} \FunctionTok{c}\NormalTok{(}\SpecialCharTok{{-}}\DecValTok{2}\NormalTok{, }\DecValTok{10}\NormalTok{)) }\SpecialCharTok{+}
    \FunctionTok{ggtitle}\NormalTok{(}\FunctionTok{paste0}\NormalTok{(}\StringTok{"ROI:"}\NormalTok{, region))}
  
\NormalTok{  plotobject}
\NormalTok{\}}
\end{Highlighting}
\end{Shaded}

\begin{Shaded}
\begin{Highlighting}[]
\FunctionTok{plot\_univar\_eachrun\_nonfocal}\NormalTok{(}\StringTok{"precentral\_A"}\NormalTok{)}
\end{Highlighting}
\end{Shaded}

\begin{verbatim}
## Warning: Using alpha for a discrete variable is not advised.
\end{verbatim}

\includegraphics[width=1\linewidth]{1_Univariate_files/figure-latex/unnamed-chunk-18-1}

\begin{Shaded}
\begin{Highlighting}[]
\FunctionTok{plot\_univar\_eachrun\_nonfocal}\NormalTok{(}\StringTok{"medialFrontal"}\NormalTok{) }\CommentTok{\#dACC}
\end{Highlighting}
\end{Shaded}

\begin{verbatim}
## Warning: Using alpha for a discrete variable is not advised.
\end{verbatim}

\includegraphics[width=1\linewidth]{1_Univariate_files/figure-latex/unnamed-chunk-18-2}

\begin{Shaded}
\begin{Highlighting}[]
\FunctionTok{plot\_univar\_eachrun\_nonfocal}\NormalTok{(}\StringTok{"midFrontal\_"}\NormalTok{) }\CommentTok{\#DLPFC}
\end{Highlighting}
\end{Shaded}

\begin{verbatim}
## Warning: Using alpha for a discrete variable is not advised.
\end{verbatim}

\includegraphics[width=1\linewidth]{1_Univariate_files/figure-latex/unnamed-chunk-18-3}

\begin{Shaded}
\begin{Highlighting}[]
\FunctionTok{plot\_univar\_eachrun\_nonfocal}\NormalTok{(}\StringTok{"midFrontalOrb"}\NormalTok{)}
\end{Highlighting}
\end{Shaded}

\begin{verbatim}
## Warning: Using alpha for a discrete variable is not advised.
\end{verbatim}

\includegraphics[width=1\linewidth]{1_Univariate_files/figure-latex/unnamed-chunk-18-4}

\begin{Shaded}
\begin{Highlighting}[]
\FunctionTok{plot\_univar\_eachrun\_nonfocal}\NormalTok{(}\StringTok{"medialFrontal"}\NormalTok{)}
\end{Highlighting}
\end{Shaded}

\begin{verbatim}
## Warning: Using alpha for a discrete variable is not advised.
\end{verbatim}

\includegraphics[width=1\linewidth]{1_Univariate_files/figure-latex/unnamed-chunk-18-5}

\begin{Shaded}
\begin{Highlighting}[]
\FunctionTok{plot\_univar\_eachrun\_nonfocal}\NormalTok{(}\StringTok{"MPFC"}\NormalTok{)}
\end{Highlighting}
\end{Shaded}

\begin{verbatim}
## Warning: Using alpha for a discrete variable is not advised.
\end{verbatim}

\includegraphics[width=1\linewidth]{1_Univariate_files/figure-latex/unnamed-chunk-18-6}

Then we repeat the same modeling procedure on these regions. But this
time, we fit an interaction between event and domain in all regions,
because some MD regions appear to have a preference for physical events,
and it is an open quesiton, if this is true, whether they too could
encode physical prediction error.

This set of regions includes left and right V1, MT, insula, and IFG
(rather than bilateral). The bilateral versions of these ROIs are not
included in this portion of the analysis.

\begin{Shaded}
\begin{Highlighting}[]
\NormalTok{ROIs }\OtherTok{\textless{}{-}} \FunctionTok{levels}\NormalTok{(}\FunctionTok{as.factor}\NormalTok{(univariate\_data\_nonfocal}\SpecialCharTok{$}\NormalTok{ROI\_name\_final))}

\NormalTok{ROI\_data\_nonfocal\_runs12 }\OtherTok{\textless{}{-}}\NormalTok{ univariate\_data\_nonfocal }\SpecialCharTok{\%\textgreater{}\%}
    \FunctionTok{filter}\NormalTok{(event }\SpecialCharTok{!=} \StringTok{"fam"}\NormalTok{,}
\NormalTok{           domain }\SpecialCharTok{!=} \StringTok{"both"}\NormalTok{,}
\NormalTok{           top\_n\_voxels }\SpecialCharTok{==} \StringTok{"100"}\NormalTok{) }\SpecialCharTok{\%\textgreater{}\%}
    \FunctionTok{filter}\NormalTok{(extracted\_run\_number }\SpecialCharTok{==} \StringTok{"run1"} \SpecialCharTok{|}\NormalTok{ extracted\_run\_number }\SpecialCharTok{==} \StringTok{"run2"}\NormalTok{)}

\NormalTok{ROI\_data\_nonfocal\_allruns }\OtherTok{\textless{}{-}}\NormalTok{ univariate\_data\_nonfocal }\SpecialCharTok{\%\textgreater{}\%}
    \FunctionTok{filter}\NormalTok{(event }\SpecialCharTok{!=} \StringTok{"fam"}\NormalTok{,}
\NormalTok{           domain }\SpecialCharTok{!=} \StringTok{"both"}\NormalTok{,}
\NormalTok{           top\_n\_voxels }\SpecialCharTok{==} \StringTok{"100"}\NormalTok{) }\SpecialCharTok{\%\textgreater{}\%}
    \FunctionTok{filter}\NormalTok{(extracted\_run\_number }\SpecialCharTok{==} \StringTok{"all\_runs"}\NormalTok{)}

\NormalTok{all\_modelsummaries\_nonfocal\_100 }\OtherTok{\textless{}{-}} \FunctionTok{data.frame}\NormalTok{()}

\ControlFlowTok{for}\NormalTok{ (ROI }\ControlFlowTok{in}\NormalTok{ ROIs) \{}
\NormalTok{  curROI }\OtherTok{\textless{}{-}}\NormalTok{ ROI}
\NormalTok{  ROIdata }\OtherTok{\textless{}{-}}\NormalTok{ ROI\_data\_nonfocal\_runs12 }\SpecialCharTok{\%\textgreater{}\%}
    \FunctionTok{filter}\NormalTok{(ROI\_name\_final }\SpecialCharTok{==}\NormalTok{ curROI)}
  \FunctionTok{paste0}\NormalTok{(}\StringTok{"Current ROI: "}\NormalTok{, curROI)}
  \ControlFlowTok{if}\NormalTok{ (ROIdata}\SpecialCharTok{$}\NormalTok{ROI\_category[}\DecValTok{1}\NormalTok{] }\SpecialCharTok{\%in\%}\NormalTok{ domain\_specific\_regions) \{}
\NormalTok{  ROI\_category }\OtherTok{\textless{}{-}} \StringTok{"specific"}  
\NormalTok{  model }\OtherTok{\textless{}{-}} \FunctionTok{lmer}\NormalTok{(}\AttributeTok{data =}\NormalTok{ ROIdata,}
     \AttributeTok{formula =}\NormalTok{ meanbeta }\SpecialCharTok{\textasciitilde{}}\NormalTok{ event }\SpecialCharTok{*}\NormalTok{ domain }\SpecialCharTok{+}\NormalTok{ (}\DecValTok{1}\SpecialCharTok{|}\NormalTok{subjectID))}
  
\NormalTok{  \} }\ControlFlowTok{else}\NormalTok{ \{}
\NormalTok{      ROI\_category }\OtherTok{\textless{}{-}} \StringTok{"general"}  
\NormalTok{    model }\OtherTok{\textless{}{-}} \FunctionTok{lmer}\NormalTok{(}\AttributeTok{data =}\NormalTok{ ROIdata,}
     \AttributeTok{formula =}\NormalTok{ meanbeta }\SpecialCharTok{\textasciitilde{}}\NormalTok{ event }\SpecialCharTok{*}\NormalTok{ domain }\SpecialCharTok{+}\NormalTok{ (}\DecValTok{1}\SpecialCharTok{|}\NormalTok{subjectID))}
  
\NormalTok{  \}}
\NormalTok{  modelsummary }\OtherTok{\textless{}{-}} \FunctionTok{cbind}\NormalTok{(ROI\_category, curROI, }\FunctionTok{rownames}\NormalTok{( }\FunctionTok{summary}\NormalTok{(model)}\SpecialCharTok{$}\NormalTok{coefficients), }\FunctionTok{summary}\NormalTok{(model)}\SpecialCharTok{$}\NormalTok{coefficients) }\SpecialCharTok{\%\textgreater{}\%}
    \FunctionTok{as.data.frame}\NormalTok{()}
  \FunctionTok{colnames}\NormalTok{(modelsummary) }\OtherTok{\textless{}{-}} \FunctionTok{c}\NormalTok{(}\StringTok{"domain"}\NormalTok{, }\StringTok{"ROI"}\NormalTok{, }\StringTok{"effect"}\NormalTok{, }\StringTok{"B"}\NormalTok{, }\StringTok{"SE"}\NormalTok{, }\StringTok{"df"}\NormalTok{, }\StringTok{"t"}\NormalTok{, }\StringTok{"p"}\NormalTok{)}
\NormalTok{  all\_modelsummaries\_nonfocal\_100 }\OtherTok{\textless{}{-}} \FunctionTok{rbind}\NormalTok{(all\_modelsummaries\_nonfocal\_100, modelsummary)}
\NormalTok{\}}


\NormalTok{all\_modelsummaries\_nonfocal\_100 }\OtherTok{\textless{}{-}}\NormalTok{ all\_modelsummaries\_nonfocal\_100 }\SpecialCharTok{\%\textgreater{}\%}
  \FunctionTok{mutate\_at}\NormalTok{(}\FunctionTok{c}\NormalTok{(}\DecValTok{4}\SpecialCharTok{:}\DecValTok{8}\NormalTok{), as.numeric) }\SpecialCharTok{\%\textgreater{}\%}
  \FunctionTok{mutate}\NormalTok{(}\AttributeTok{sig =} \FunctionTok{as.factor}\NormalTok{(}\FunctionTok{case\_when}\NormalTok{(p }\SpecialCharTok{\textless{}}\NormalTok{ .}\DecValTok{05} \SpecialCharTok{\textasciitilde{}} \StringTok{"yes"}\NormalTok{,}
\NormalTok{                                   p }\SpecialCharTok{\textgreater{}=}\NormalTok{ .}\DecValTok{05} \SpecialCharTok{\textasciitilde{}} \StringTok{"no"}\NormalTok{)),}
         \AttributeTok{star =} \FunctionTok{as.factor}\NormalTok{(}
           \FunctionTok{case\_when}\NormalTok{(p }\SpecialCharTok{\textless{}}\NormalTok{ .}\DecValTok{001} \SpecialCharTok{\textasciitilde{}} \StringTok{"***"}\NormalTok{,}
\NormalTok{                     p }\SpecialCharTok{\textless{}}\NormalTok{ .}\DecValTok{01} \SpecialCharTok{\textasciitilde{}} \StringTok{"**"}\NormalTok{,}
\NormalTok{                     p }\SpecialCharTok{\textless{}}\NormalTok{ .}\DecValTok{05} \SpecialCharTok{\textasciitilde{}} \StringTok{"*"}\NormalTok{,}
\NormalTok{                     p }\SpecialCharTok{\textless{}}\NormalTok{ .}\DecValTok{1} \SpecialCharTok{\textasciitilde{}} \StringTok{"\textasciitilde{}"}\NormalTok{,}
                     \ConstantTok{TRUE} \SpecialCharTok{\textasciitilde{}} \StringTok{" "}\NormalTok{)}
\NormalTok{         )) }\SpecialCharTok{\%\textgreater{}\%}
  \FunctionTok{mutate}\NormalTok{(}\AttributeTok{p =} \FunctionTok{round}\NormalTok{(p, }\AttributeTok{digits =} \DecValTok{3}\NormalTok{))}

\NormalTok{all\_modelsummaries\_nonfocal\_100}\SpecialCharTok{$}\NormalTok{effect }\OtherTok{\textless{}{-}}
  \FunctionTok{as.factor}\NormalTok{(all\_modelsummaries\_nonfocal\_100}\SpecialCharTok{$}\NormalTok{effect)}

\FunctionTok{levels}\NormalTok{(all\_modelsummaries\_nonfocal\_100}\SpecialCharTok{$}\NormalTok{effect) }\OtherTok{\textless{}{-}}
  \FunctionTok{c}\NormalTok{(}\StringTok{"intercept"}\NormalTok{, }\StringTok{"domain"}\NormalTok{, }\StringTok{"event"}\NormalTok{, }\StringTok{"event:domain"}\NormalTok{)}

\FunctionTok{rownames}\NormalTok{(all\_modelsummaries\_nonfocal\_100) }\OtherTok{\textless{}{-}} \ConstantTok{NULL}

\NormalTok{knitr}\SpecialCharTok{::}\FunctionTok{kable}\NormalTok{(dplyr}\SpecialCharTok{::}\FunctionTok{arrange}\NormalTok{(all\_modelsummaries\_nonfocal\_100, effect), }\AttributeTok{digits =} \DecValTok{3}\NormalTok{)}
\end{Highlighting}
\end{Shaded}

\begin{longtable}[]{@{}
  >{\raggedright\arraybackslash}p{(\columnwidth - 18\tabcolsep) * \real{0.1059}}
  >{\raggedright\arraybackslash}p{(\columnwidth - 18\tabcolsep) * \real{0.2824}}
  >{\raggedright\arraybackslash}p{(\columnwidth - 18\tabcolsep) * \real{0.1529}}
  >{\raggedleft\arraybackslash}p{(\columnwidth - 18\tabcolsep) * \real{0.0824}}
  >{\raggedleft\arraybackslash}p{(\columnwidth - 18\tabcolsep) * \real{0.0706}}
  >{\raggedleft\arraybackslash}p{(\columnwidth - 18\tabcolsep) * \real{0.0471}}
  >{\raggedleft\arraybackslash}p{(\columnwidth - 18\tabcolsep) * \real{0.0824}}
  >{\raggedleft\arraybackslash}p{(\columnwidth - 18\tabcolsep) * \real{0.0706}}
  >{\raggedright\arraybackslash}p{(\columnwidth - 18\tabcolsep) * \real{0.0471}}
  >{\raggedright\arraybackslash}p{(\columnwidth - 18\tabcolsep) * \real{0.0588}}@{}}
\toprule
\begin{minipage}[b]{\linewidth}\raggedright
domain
\end{minipage} & \begin{minipage}[b]{\linewidth}\raggedright
ROI
\end{minipage} & \begin{minipage}[b]{\linewidth}\raggedright
effect
\end{minipage} & \begin{minipage}[b]{\linewidth}\raggedleft
B
\end{minipage} & \begin{minipage}[b]{\linewidth}\raggedleft
SE
\end{minipage} & \begin{minipage}[b]{\linewidth}\raggedleft
df
\end{minipage} & \begin{minipage}[b]{\linewidth}\raggedleft
t
\end{minipage} & \begin{minipage}[b]{\linewidth}\raggedleft
p
\end{minipage} & \begin{minipage}[b]{\linewidth}\raggedright
sig
\end{minipage} & \begin{minipage}[b]{\linewidth}\raggedright
star
\end{minipage} \\
\midrule
\endhead
general & antParietal\_L & intercept & 2.756 & 0.328 & 31 & 8.401 &
0.000 & yes & *** \\
general & antParietal\_R & intercept & 3.746 & 0.391 & 31 & 9.570 &
0.000 & yes & *** \\
general & insula\_L & intercept & 0.642 & 0.171 & 31 & 3.749 & 0.001 &
yes & *** \\
general & insula\_R & intercept & 0.687 & 0.157 & 31 & 4.376 & 0.000 &
yes & *** \\
general & medialFrontal\_L & intercept & 1.255 & 0.211 & 31 & 5.946 &
0.000 & yes & *** \\
general & medialFrontal\_R & intercept & 1.248 & 0.203 & 31 & 6.151 &
0.000 & yes & *** \\
general & midFrontal\_L & intercept & 1.887 & 0.315 & 31 & 5.989 & 0.000
& yes & *** \\
general & midFrontal\_R & intercept & 2.194 & 0.298 & 31 & 7.352 & 0.000
& yes & *** \\
general & midFrontalOrb\_L & intercept & 0.764 & 0.304 & 31 & 2.514 &
0.017 & yes & * \\
general & midFrontalOrb\_R & intercept & 0.994 & 0.277 & 31 & 3.592 &
0.001 & yes & ** \\
general & midParietal\_L & intercept & 4.142 & 0.566 & 31 & 7.317 &
0.000 & yes & *** \\
general & midParietal\_R & intercept & 5.468 & 0.544 & 31 & 10.046 &
0.000 & yes & *** \\
specific & MPFC\_L & intercept & -0.725 & 0.300 & 31 & -2.420 & 0.022 &
yes & * \\
specific & MPFC\_R & intercept & -0.694 & 0.238 & 31 & -2.912 & 0.007 &
yes & ** \\
general & MT\_L & intercept & 4.897 & 0.502 & 31 & 9.757 & 0.000 & yes &
*** \\
general & MT\_R & intercept & 5.218 & 0.408 & 31 & 12.788 & 0.000 & yes
& *** \\
specific & parahip-gyrus\_L & intercept & 0.008 & 0.054 & 31 & 0.156 &
0.877 & no & \\
specific & parahip-gyrus\_R & intercept & 0.143 & 0.066 & 31 & 2.171 &
0.038 & yes & * \\
general & postParietal\_L & intercept & 6.250 & 0.797 & 31 & 7.838 &
0.000 & yes & *** \\
general & postParietal\_R & intercept & 8.116 & 0.921 & 31 & 8.814 &
0.000 & yes & *** \\
general & precentral\_A\_preCG\_L & intercept & 2.946 & 0.266 & 31 &
11.061 & 0.000 & yes & *** \\
general & precentral\_A\_preCG\_R & intercept & 3.896 & 0.307 & 31 &
12.706 & 0.000 & yes & *** \\
general & precentral\_B\_IFGop\_L & intercept & 2.129 & 0.224 & 31 &
9.519 & 0.000 & yes & *** \\
general & precentral\_B\_IFGop\_R & intercept & 2.869 & 0.274 & 31 &
10.484 & 0.000 & yes & *** \\
specific & precentral-supfrontal\_L & intercept & 1.822 & 0.255 & 31 &
7.154 & 0.000 & yes & *** \\
specific & precentral-supfrontal\_R & intercept & 2.067 & 0.268 & 31 &
7.703 & 0.000 & yes & *** \\
general & supFrontal\_L & intercept & 2.273 & 0.296 & 31 & 7.691 & 0.000
& yes & *** \\
general & supFrontal\_R & intercept & 2.925 & 0.255 & 31 & 11.455 &
0.000 & yes & *** \\
specific & supinffrontal\_L & intercept & 0.769 & 0.221 & 31 & 3.482 &
0.002 & yes & ** \\
specific & supinffrontal\_R & intercept & 1.487 & 0.224 & 31 & 6.628 &
0.000 & yes & *** \\
specific & supparietal\_L & intercept & 5.142 & 0.639 & 31 & 8.041 &
0.000 & yes & *** \\
specific & supparietal\_R & intercept & 5.252 & 0.781 & 31 & 6.723 &
0.000 & yes & *** \\
general & V1\_L & intercept & 8.297 & 0.556 & 31 & 14.915 & 0.000 & yes
& *** \\
general & V1\_R & intercept & 9.154 & 0.718 & 31 & 12.741 & 0.000 & yes
& *** \\
specific & vislateralventral\_L & intercept & 3.159 & 0.386 & 31 & 8.180
& 0.000 & yes & *** \\
specific & vislateralventral\_R & intercept & 5.986 & 0.678 & 31 & 8.832
& 0.000 & yes & *** \\
specific & vismedial\_L & intercept & 3.193 & 0.500 & 31 & 6.391 & 0.000
& yes & *** \\
specific & vismedial\_R & intercept & 2.653 & 0.416 & 31 & 6.381 & 0.000
& yes & *** \\
general & antParietal\_L & domain & 0.149 & 0.081 & 221 & 1.838 & 0.067
& no & \textasciitilde{} \\
general & antParietal\_R & domain & 0.733 & 0.111 & 221 & 6.584 & 0.000
& yes & *** \\
general & insula\_L & domain & -0.051 & 0.054 & 221 & -0.933 & 0.352 &
no & \\
general & insula\_R & domain & 0.044 & 0.048 & 221 & 0.929 & 0.354 & no
& \\
general & medialFrontal\_L & domain & -0.078 & 0.063 & 221 & -1.246 &
0.214 & no & \\
general & medialFrontal\_R & domain & 0.061 & 0.071 & 221 & 0.859 &
0.391 & no & \\
general & midFrontal\_L & domain & -0.116 & 0.099 & 221 & -1.170 & 0.243
& no & \\
general & midFrontal\_R & domain & 0.259 & 0.104 & 221 & 2.496 & 0.013 &
yes & * \\
general & midFrontalOrb\_L & domain & -0.179 & 0.114 & 221 & -1.578 &
0.116 & no & \\
general & midFrontalOrb\_R & domain & 0.085 & 0.104 & 221 & 0.819 &
0.414 & no & \\
general & midParietal\_L & domain & 0.201 & 0.108 & 221 & 1.857 & 0.065
& no & \textasciitilde{} \\
general & midParietal\_R & domain & 0.905 & 0.125 & 221 & 7.220 & 0.000
& yes & *** \\
specific & MPFC\_L & domain & -0.481 & 0.123 & 221 & -3.920 & 0.000 &
yes & *** \\
specific & MPFC\_R & domain & -0.235 & 0.114 & 221 & -2.056 & 0.041 &
yes & * \\
general & MT\_L & domain & 0.539 & 0.077 & 221 & 6.989 & 0.000 & yes &
*** \\
general & MT\_R & domain & 0.744 & 0.068 & 221 & 10.917 & 0.000 & yes &
*** \\
specific & parahip-gyrus\_L & domain & -0.031 & 0.027 & 221 & -1.161 &
0.247 & no & \\
specific & parahip-gyrus\_R & domain & -0.077 & 0.026 & 221 & -2.948 &
0.004 & yes & ** \\
general & postParietal\_L & domain & 0.379 & 0.118 & 221 & 3.214 & 0.002
& yes & ** \\
general & postParietal\_R & domain & 1.150 & 0.141 & 221 & 8.141 & 0.000
& yes & *** \\
general & precentral\_A\_preCG\_L & domain & 0.077 & 0.098 & 221 & 0.790
& 0.430 & no & \\
general & precentral\_A\_preCG\_R & domain & 0.593 & 0.097 & 221 & 6.120
& 0.000 & yes & *** \\
general & precentral\_B\_IFGop\_L & domain & 0.041 & 0.088 & 221 & 0.468
& 0.640 & no & \\
general & precentral\_B\_IFGop\_R & domain & 0.459 & 0.088 & 221 & 5.201
& 0.000 & yes & *** \\
specific & precentral-supfrontal\_L & domain & 0.068 & 0.055 & 221 &
1.235 & 0.218 & no & \\
specific & precentral-supfrontal\_R & domain & 0.362 & 0.059 & 221 &
6.138 & 0.000 & yes & *** \\
general & supFrontal\_L & domain & 0.081 & 0.071 & 221 & 1.141 & 0.255 &
no & \\
general & supFrontal\_R & domain & 0.478 & 0.078 & 221 & 6.116 & 0.000 &
yes & *** \\
specific & supinffrontal\_L & domain & -0.211 & 0.085 & 221 & -2.473 &
0.014 & yes & * \\
specific & supinffrontal\_R & domain & 0.106 & 0.089 & 221 & 1.181 &
0.239 & no & \\
specific & supparietal\_L & domain & 0.456 & 0.108 & 221 & 4.207 & 0.000
& yes & *** \\
specific & supparietal\_R & domain & 0.912 & 0.128 & 221 & 7.150 & 0.000
& yes & *** \\
general & V1\_L & domain & 0.726 & 0.125 & 221 & 5.788 & 0.000 & yes &
*** \\
general & V1\_R & domain & 0.351 & 0.134 & 221 & 2.617 & 0.009 & yes &
** \\
specific & vislateralventral\_L & domain & -0.554 & 0.092 & 221 & -6.033
& 0.000 & yes & *** \\
specific & vislateralventral\_R & domain & -0.313 & 0.132 & 221 & -2.377
& 0.018 & yes & * \\
specific & vismedial\_L & domain & 0.884 & 0.102 & 221 & 8.658 & 0.000 &
yes & *** \\
specific & vismedial\_R & domain & 1.073 & 0.090 & 221 & 11.962 & 0.000
& yes & *** \\
general & antParietal\_L & event & -0.149 & 0.081 & 221 & -1.839 & 0.067
& no & \textasciitilde{} \\
general & antParietal\_R & event & -0.251 & 0.111 & 221 & -2.252 & 0.025
& yes & * \\
general & insula\_L & event & -0.029 & 0.054 & 221 & -0.537 & 0.592 & no
& \\
general & insula\_R & event & -0.066 & 0.048 & 221 & -1.383 & 0.168 & no
& \\
general & medialFrontal\_L & event & -0.041 & 0.063 & 221 & -0.649 &
0.517 & no & \\
general & medialFrontal\_R & event & -0.056 & 0.071 & 221 & -0.786 &
0.433 & no & \\
general & midFrontal\_L & event & -0.014 & 0.099 & 221 & -0.137 & 0.891
& no & \\
general & midFrontal\_R & event & -0.124 & 0.104 & 221 & -1.197 & 0.233
& no & \\
general & midFrontalOrb\_L & event & 0.130 & 0.114 & 221 & 1.147 & 0.253
& no & \\
general & midFrontalOrb\_R & event & 0.024 & 0.104 & 221 & 0.233 & 0.816
& no & \\
general & midParietal\_L & event & -0.085 & 0.108 & 221 & -0.782 & 0.435
& no & \\
general & midParietal\_R & event & -0.317 & 0.125 & 221 & -2.525 & 0.012
& yes & * \\
specific & MPFC\_L & event & -0.112 & 0.123 & 221 & -0.914 & 0.362 & no
& \\
specific & MPFC\_R & event & 0.018 & 0.114 & 221 & 0.155 & 0.877 & no
& \\
general & MT\_L & event & -0.019 & 0.077 & 221 & -0.252 & 0.801 & no
& \\
general & MT\_R & event & -0.097 & 0.068 & 221 & -1.428 & 0.155 & no
& \\
specific & parahip-gyrus\_L & event & -0.038 & 0.027 & 221 & -1.413 &
0.159 & no & \\
specific & parahip-gyrus\_R & event & -0.044 & 0.026 & 221 & -1.671 &
0.096 & no & \textasciitilde{} \\
general & postParietal\_L & event & -0.153 & 0.118 & 221 & -1.296 &
0.196 & no & \\
general & postParietal\_R & event & -0.221 & 0.141 & 221 & -1.566 &
0.119 & no & \\
general & precentral\_A\_preCG\_L & event & -0.165 & 0.098 & 221 &
-1.683 & 0.094 & no & \textasciitilde{} \\
general & precentral\_A\_preCG\_R & event & -0.256 & 0.097 & 221 &
-2.641 & 0.009 & yes & ** \\
general & precentral\_B\_IFGop\_L & event & -0.118 & 0.088 & 221 &
-1.334 & 0.183 & no & \\
general & precentral\_B\_IFGop\_R & event & -0.217 & 0.088 & 221 &
-2.465 & 0.014 & yes & * \\
specific & precentral-supfrontal\_L & event & -0.002 & 0.055 & 221 &
-0.029 & 0.977 & no & \\
specific & precentral-supfrontal\_R & event & -0.076 & 0.059 & 221 &
-1.294 & 0.197 & no & \\
general & supFrontal\_L & event & -0.065 & 0.071 & 221 & -0.910 & 0.364
& no & \\
general & supFrontal\_R & event & -0.183 & 0.078 & 221 & -2.339 & 0.020
& yes & * \\
specific & supinffrontal\_L & event & -0.041 & 0.085 & 221 & -0.482 &
0.630 & no & \\
specific & supinffrontal\_R & event & -0.104 & 0.089 & 221 & -1.164 &
0.246 & no & \\
specific & supparietal\_L & event & -0.099 & 0.108 & 221 & -0.913 &
0.362 & no & \\
specific & supparietal\_R & event & -0.156 & 0.128 & 221 & -1.222 &
0.223 & no & \\
general & V1\_L & event & -0.026 & 0.125 & 221 & -0.207 & 0.836 & no
& \\
general & V1\_R & event & -0.069 & 0.134 & 221 & -0.514 & 0.608 & no
& \\
specific & vislateralventral\_L & event & -0.106 & 0.092 & 221 & -1.156
& 0.249 & no & \\
specific & vislateralventral\_R & event & -0.073 & 0.132 & 221 & -0.552
& 0.582 & no & \\
specific & vismedial\_L & event & 0.010 & 0.102 & 221 & 0.096 & 0.924 &
no & \\
specific & vismedial\_R & event & -0.128 & 0.090 & 221 & -1.431 & 0.154
& no & \\
general & antParietal\_L & event:domain & -0.033 & 0.081 & 221 & -0.409
& 0.683 & no & \\
general & antParietal\_R & event:domain & 0.124 & 0.111 & 221 & 1.117 &
0.265 & no & \\
general & insula\_L & event:domain & 0.018 & 0.054 & 221 & 0.330 & 0.742
& no & \\
general & insula\_R & event:domain & 0.044 & 0.048 & 221 & 0.927 & 0.355
& no & \\
general & medialFrontal\_L & event:domain & 0.051 & 0.063 & 221 & 0.812
& 0.418 & no & \\
general & medialFrontal\_R & event:domain & 0.134 & 0.071 & 221 & 1.897
& 0.059 & no & \textasciitilde{} \\
general & midFrontal\_L & event:domain & 0.109 & 0.099 & 221 & 1.096 &
0.274 & no & \\
general & midFrontal\_R & event:domain & 0.204 & 0.104 & 221 & 1.971 &
0.050 & yes & * \\
general & midFrontalOrb\_L & event:domain & 0.135 & 0.114 & 221 & 1.187
& 0.237 & no & \\
general & midFrontalOrb\_R & event:domain & 0.130 & 0.104 & 221 & 1.253
& 0.212 & no & \\
general & midParietal\_L & event:domain & 0.088 & 0.108 & 221 & 0.808 &
0.420 & no & \\
general & midParietal\_R & event:domain & 0.181 & 0.125 & 221 & 1.440 &
0.151 & no & \\
specific & MPFC\_L & event:domain & -0.164 & 0.123 & 221 & -1.334 &
0.183 & no & \\
specific & MPFC\_R & event:domain & -0.142 & 0.114 & 221 & -1.240 &
0.216 & no & \\
general & MT\_L & event:domain & 0.013 & 0.077 & 221 & 0.169 & 0.866 &
no & \\
general & MT\_R & event:domain & 0.001 & 0.068 & 221 & 0.009 & 0.993 &
no & \\
specific & parahip-gyrus\_L & event:domain & 0.008 & 0.027 & 221 & 0.302
& 0.763 & no & \\
specific & parahip-gyrus\_R & event:domain & -0.013 & 0.026 & 221 &
-0.479 & 0.632 & no & \\
general & postParietal\_L & event:domain & 0.192 & 0.118 & 221 & 1.633 &
0.104 & no & \\
general & postParietal\_R & event:domain & 0.189 & 0.141 & 221 & 1.336 &
0.183 & no & \\
general & precentral\_A\_preCG\_L & event:domain & 0.057 & 0.098 & 221 &
0.588 & 0.557 & no & \\
general & precentral\_A\_preCG\_R & event:domain & 0.122 & 0.097 & 221 &
1.260 & 0.209 & no & \\
general & precentral\_B\_IFGop\_L & event:domain & 0.080 & 0.088 & 221 &
0.903 & 0.368 & no & \\
general & precentral\_B\_IFGop\_R & event:domain & 0.022 & 0.088 & 221 &
0.251 & 0.802 & no & \\
specific & precentral-supfrontal\_L & event:domain & -0.037 & 0.055 &
221 & -0.668 & 0.505 & no & \\
specific & precentral-supfrontal\_R & event:domain & -0.015 & 0.059 &
221 & -0.257 & 0.798 & no & \\
general & supFrontal\_L & event:domain & 0.071 & 0.071 & 221 & 0.995 &
0.321 & no & \\
general & supFrontal\_R & event:domain & 0.086 & 0.078 & 221 & 1.098 &
0.273 & no & \\
specific & supinffrontal\_L & event:domain & 0.051 & 0.085 & 221 & 0.602
& 0.547 & no & \\
specific & supinffrontal\_R & event:domain & 0.027 & 0.089 & 221 & 0.301
& 0.764 & no & \\
specific & supparietal\_L & event:domain & -0.025 & 0.108 & 221 & -0.231
& 0.817 & no & \\
specific & supparietal\_R & event:domain & 0.052 & 0.128 & 221 & 0.407 &
0.685 & no & \\
general & V1\_L & event:domain & -0.018 & 0.125 & 221 & -0.140 & 0.889 &
no & \\
general & V1\_R & event:domain & -0.063 & 0.134 & 221 & -0.470 & 0.639 &
no & \\
specific & vislateralventral\_L & event:domain & 0.096 & 0.092 & 221 &
1.040 & 0.299 & no & \\
specific & vislateralventral\_R & event:domain & -0.011 & 0.132 & 221 &
-0.083 & 0.934 & no & \\
specific & vismedial\_L & event:domain & -0.069 & 0.102 & 221 & -0.673 &
0.502 & no & \\
specific & vismedial\_R & event:domain & 0.107 & 0.090 & 221 & 1.190 &
0.235 & no & \\
\bottomrule
\end{longtable}

For each region for which we see a significant interaction, we unpack it
and test whether it provides a better fit than a model with just the
main effects.

\begin{Shaded}
\begin{Highlighting}[]
\NormalTok{medF\_R\_model }\OtherTok{\textless{}{-}} \FunctionTok{lmer}\NormalTok{(}
      \AttributeTok{data =}\NormalTok{ ROI\_data\_nonfocal\_runs12 }\SpecialCharTok{\%\textgreater{}\%}
        \FunctionTok{filter}\NormalTok{(ROI\_name\_final }\SpecialCharTok{==} \StringTok{"medialFrontal\_R"}\NormalTok{),}
      \AttributeTok{formula =}\NormalTok{ meanbeta }\SpecialCharTok{\textasciitilde{}}\NormalTok{ event }\SpecialCharTok{*}\NormalTok{ domain }\SpecialCharTok{+}\NormalTok{ (}\DecValTok{1}\SpecialCharTok{|}\NormalTok{extracted\_run\_number) }\SpecialCharTok{+}\NormalTok{ (}\DecValTok{1}\SpecialCharTok{|}\NormalTok{subjectID)}
\NormalTok{    )}
\end{Highlighting}
\end{Shaded}

\begin{verbatim}
## boundary (singular) fit: see ?isSingular
\end{verbatim}

\begin{Shaded}
\begin{Highlighting}[]
\CommentTok{\# summary(medF\_R\_model)}
\NormalTok{medF\_R\_model }\SpecialCharTok{\%\textgreater{}\%}
  \FunctionTok{apa\_print}\NormalTok{() }\SpecialCharTok{\%\textgreater{}\%}
  \FunctionTok{apa\_table}\NormalTok{()}
\end{Highlighting}
\end{Shaded}

\begin{table}[tbp]

\begin{center}
\begin{threeparttable}

\caption{\label{tab:unnamed-chunk-20}}

\begin{tabular}{llllll}
\toprule
Term & \multicolumn{1}{c}{$\hat{\beta}$} & \multicolumn{1}{c}{95\% CI} & \multicolumn{1}{c}{$t$} & \multicolumn{1}{c}{$\mathit{df}$} & \multicolumn{1}{c}{$p$}\\
\midrule
Intercept & 1.25 & {}[0.85, 1.65] & 6.15 & 31.00 & < .001\\
Event1 & -0.06 & {}[-0.19, 0.08] & -0.79 & 221.00 & .433\\
Domain1 & 0.06 & {}[-0.08, 0.20] & 0.86 & 221.00 & .391\\
Event1 $\times$ Domain1 & 0.13 & {}[0.00, 0.27] & 1.90 & 221.00 & .059\\
\bottomrule
\end{tabular}

\end{threeparttable}
\end{center}

\end{table}

\begin{Shaded}
\begin{Highlighting}[]
\FunctionTok{plot}\NormalTok{(}\FunctionTok{allEffects}\NormalTok{(medF\_R\_model))}
\end{Highlighting}
\end{Shaded}

\includegraphics[width=0.5\linewidth]{1_Univariate_files/figure-latex/unnamed-chunk-20-1}

\begin{Shaded}
\begin{Highlighting}[]
\CommentTok{\# check\_model(medF\_R\_model)}
\FunctionTok{check\_outliers}\NormalTok{(medF\_R\_model)}
\end{Highlighting}
\end{Shaded}

\begin{verbatim}
## OK: No outliers detected.
\end{verbatim}

\begin{Shaded}
\begin{Highlighting}[]
\FunctionTok{lsmeans}\NormalTok{(medF\_R\_model, pairwise }\SpecialCharTok{\textasciitilde{}}\NormalTok{ event }\SpecialCharTok{|}\NormalTok{ domain)}\SpecialCharTok{$}\NormalTok{contrasts}
\end{Highlighting}
\end{Shaded}

\begin{verbatim}
## domain = physics:
##  contrast    estimate  SE  df t.ratio p.value
##  exp - unexp    0.157 0.2 220  0.786  0.4330 
## 
## domain = psychology:
##  contrast    estimate  SE  df t.ratio p.value
##  exp - unexp   -0.380 0.2 220 -1.898  0.0590 
## 
## Degrees-of-freedom method: kenward-roger
\end{verbatim}

\begin{Shaded}
\begin{Highlighting}[]
\NormalTok{medF\_R\_model\_simpler }\OtherTok{\textless{}{-}} \FunctionTok{lmer}\NormalTok{(}\AttributeTok{data =}\NormalTok{ ROI\_data\_nonfocal\_runs12 }\SpecialCharTok{\%\textgreater{}\%}
        \FunctionTok{filter}\NormalTok{(ROI\_name\_final }\SpecialCharTok{==} \StringTok{"medialFrontal\_R"}\NormalTok{),}
      \AttributeTok{formula =}\NormalTok{ meanbeta }\SpecialCharTok{\textasciitilde{}}\NormalTok{ event }\SpecialCharTok{+}\NormalTok{ domain }\SpecialCharTok{+}\NormalTok{ (}\DecValTok{1}\SpecialCharTok{|}\NormalTok{extracted\_run\_number) }\SpecialCharTok{+}\NormalTok{ (}\DecValTok{1}\SpecialCharTok{|}\NormalTok{subjectID)}
\NormalTok{    )}
\end{Highlighting}
\end{Shaded}

\begin{verbatim}
## boundary (singular) fit: see ?isSingular
\end{verbatim}

\begin{Shaded}
\begin{Highlighting}[]
\FunctionTok{anova}\NormalTok{(medF\_R\_model\_simpler, medF\_R\_model)}
\end{Highlighting}
\end{Shaded}

\begin{verbatim}
## refitting model(s) with ML (instead of REML)
\end{verbatim}

\begin{verbatim}
## Data: ROI_data_nonfocal_runs12 %>% filter(ROI_name_final == "medialFrontal_R")
## Models:
## medF_R_model_simpler: meanbeta ~ event + domain + (1 | extracted_run_number) + (1 | 
## medF_R_model_simpler:     subjectID)
## medF_R_model: meanbeta ~ event * domain + (1 | extracted_run_number) + (1 | 
## medF_R_model:     subjectID)
##                      npar AIC BIC logLik deviance Chisq Df Pr(>Chisq)  
## medF_R_model_simpler    6 869 891   -429      857                      
## medF_R_model            7 868 892   -427      854  3.62  1      0.057 .
## ---
## Signif. codes:  0 '***' 0.001 '**' 0.01 '*' 0.05 '.' 0.1 ' ' 1
\end{verbatim}

\begin{Shaded}
\begin{Highlighting}[]
\NormalTok{midF\_R\_model }\OtherTok{\textless{}{-}} \FunctionTok{lmer}\NormalTok{(}
      \AttributeTok{data =}\NormalTok{ ROI\_data\_nonfocal\_runs12 }\SpecialCharTok{\%\textgreater{}\%}
        \FunctionTok{filter}\NormalTok{(ROI\_name\_final }\SpecialCharTok{==} \StringTok{"midFrontal\_R"}\NormalTok{),}
      \AttributeTok{formula =}\NormalTok{ meanbeta }\SpecialCharTok{\textasciitilde{}}\NormalTok{ event }\SpecialCharTok{*}\NormalTok{ domain }\SpecialCharTok{+}\NormalTok{ (}\DecValTok{1}\SpecialCharTok{|}\NormalTok{extracted\_run\_number) }\SpecialCharTok{+}\NormalTok{ (}\DecValTok{1}\SpecialCharTok{|}\NormalTok{subjectID)}
\NormalTok{    )}
\CommentTok{\# summary(midF\_R\_model)}
\NormalTok{midF\_R\_model }\SpecialCharTok{\%\textgreater{}\%}
  \FunctionTok{apa\_print}\NormalTok{() }\SpecialCharTok{\%\textgreater{}\%}
  \FunctionTok{apa\_table}\NormalTok{()}
\end{Highlighting}
\end{Shaded}

\begin{table}[tbp]

\begin{center}
\begin{threeparttable}

\caption{\label{tab:unnamed-chunk-21}}

\begin{tabular}{llllll}
\toprule
Term & \multicolumn{1}{c}{$\hat{\beta}$} & \multicolumn{1}{c}{95\% CI} & \multicolumn{1}{c}{$t$} & \multicolumn{1}{c}{$\mathit{df}$} & \multicolumn{1}{c}{$p$}\\
\midrule
Intercept & 2.19 & {}[1.51, 2.88] & 6.28 & 6.91 & < .001\\
Event1 & -0.12 & {}[-0.33, 0.08] & -1.21 & 220.00 & .229\\
Domain1 & 0.26 & {}[0.06, 0.46] & 2.51 & 220.00 & .013\\
Event1 $\times$ Domain1 & 0.20 & {}[0.00, 0.41] & 1.98 & 220.00 & .048\\
\bottomrule
\end{tabular}

\end{threeparttable}
\end{center}

\end{table}

\begin{Shaded}
\begin{Highlighting}[]
\FunctionTok{plot}\NormalTok{(}\FunctionTok{allEffects}\NormalTok{(midF\_R\_model))}
\end{Highlighting}
\end{Shaded}

\includegraphics[width=0.5\linewidth]{1_Univariate_files/figure-latex/unnamed-chunk-21-1}

\begin{Shaded}
\begin{Highlighting}[]
\CommentTok{\# check\_model(midF\_R\_model)}
\FunctionTok{check\_outliers}\NormalTok{(midF\_R\_model)}
\end{Highlighting}
\end{Shaded}

\begin{verbatim}
## OK: No outliers detected.
\end{verbatim}

\begin{Shaded}
\begin{Highlighting}[]
\FunctionTok{lsmeans}\NormalTok{(midF\_R\_model, pairwise }\SpecialCharTok{\textasciitilde{}}\NormalTok{ event }\SpecialCharTok{|}\NormalTok{ domain)}\SpecialCharTok{$}\NormalTok{contrasts}
\end{Highlighting}
\end{Shaded}

\begin{verbatim}
## domain = physics:
##  contrast    estimate    SE  df t.ratio p.value
##  exp - unexp    0.160 0.291 220  0.551  0.5820 
## 
## domain = psychology:
##  contrast    estimate    SE  df t.ratio p.value
##  exp - unexp   -0.657 0.291 220 -2.256  0.0250 
## 
## Degrees-of-freedom method: kenward-roger
\end{verbatim}

\begin{Shaded}
\begin{Highlighting}[]
\NormalTok{midF\_R\_model\_simpler }\OtherTok{\textless{}{-}} \FunctionTok{lmer}\NormalTok{(}
      \AttributeTok{data =}\NormalTok{ ROI\_data\_nonfocal\_runs12 }\SpecialCharTok{\%\textgreater{}\%}
        \FunctionTok{filter}\NormalTok{(ROI\_name\_final }\SpecialCharTok{==} \StringTok{"midFrontal\_R"}\NormalTok{),}
      \AttributeTok{formula =}\NormalTok{ meanbeta }\SpecialCharTok{\textasciitilde{}}\NormalTok{ event }\SpecialCharTok{+}\NormalTok{ domain }\SpecialCharTok{+}\NormalTok{ (}\DecValTok{1}\SpecialCharTok{|}\NormalTok{extracted\_run\_number) }\SpecialCharTok{+}\NormalTok{ (}\DecValTok{1}\SpecialCharTok{|}\NormalTok{subjectID)}
\NormalTok{    )}
\FunctionTok{anova}\NormalTok{(midF\_R\_model\_simpler, midF\_R\_model)}
\end{Highlighting}
\end{Shaded}

\begin{verbatim}
## refitting model(s) with ML (instead of REML)
\end{verbatim}

\begin{verbatim}
## Data: ROI_data_nonfocal_runs12 %>% filter(ROI_name_final == "midFrontal_R")
## Models:
## midF_R_model_simpler: meanbeta ~ event + domain + (1 | extracted_run_number) + (1 | 
## midF_R_model_simpler:     subjectID)
## midF_R_model: meanbeta ~ event * domain + (1 | extracted_run_number) + (1 | 
## midF_R_model:     subjectID)
##                      npar  AIC  BIC logLik deviance Chisq Df Pr(>Chisq)  
## midF_R_model_simpler    6 1064 1085   -526     1052                      
## midF_R_model            7 1062 1086   -524     1048  3.96  1      0.047 *
## ---
## Signif. codes:  0 '***' 0.001 '**' 0.01 '*' 0.05 '.' 0.1 ' ' 1
\end{verbatim}

Below we test whether the following four regions that responded more to
physical events than psychological events on the right, but not the left
(or only marginally on the left), show evidence for lateraliztion
(i.e.~a hemisphere x domain effect).

\begin{Shaded}
\begin{Highlighting}[]
\NormalTok{right\_lat\_domain\_regions }\OtherTok{\textless{}{-}} \FunctionTok{c}\NormalTok{(}\StringTok{"precentral\_A\_preCG"}\NormalTok{, }
                       \StringTok{"precentral\_B\_IFGop"}\NormalTok{,}
                       \StringTok{"supFrontal"}\NormalTok{,}
                       \StringTok{"midFrontal"}\NormalTok{)}

\NormalTok{ROI\_data\_nonfocal\_runs12\_bilat }\OtherTok{\textless{}{-}}\NormalTok{ ROI\_data\_nonfocal\_runs12 }\SpecialCharTok{\%\textgreater{}\%}
  \FunctionTok{mutate}\NormalTok{(}\AttributeTok{ROI\_LR =} \FunctionTok{as.factor}\NormalTok{(}\FunctionTok{toupper}\NormalTok{(}\FunctionTok{str\_sub}\NormalTok{(ROI\_name\_final, }\SpecialCharTok{{-}}\DecValTok{1}\NormalTok{, }\SpecialCharTok{{-}}\DecValTok{1}\NormalTok{))),}
         \AttributeTok{ROI\_bilat\_name =} \FunctionTok{as.factor}\NormalTok{(}\FunctionTok{str\_sub}\NormalTok{(ROI\_name\_final, }\DecValTok{1}\NormalTok{, }\SpecialCharTok{{-}}\DecValTok{3}\NormalTok{)))}

\NormalTok{rightlat\_domain\_model }\OtherTok{\textless{}{-}} \FunctionTok{lmer}\NormalTok{(}
      \AttributeTok{data =}\NormalTok{ ROI\_data\_nonfocal\_runs12\_bilat }\SpecialCharTok{\%\textgreater{}\%}
        \FunctionTok{filter}\NormalTok{(ROI\_bilat\_name }\SpecialCharTok{\%in\%}\NormalTok{ right\_lat\_domain\_regions),}
      \AttributeTok{formula =}\NormalTok{ meanbeta }\SpecialCharTok{\textasciitilde{}}\NormalTok{ domain }\SpecialCharTok{*}\NormalTok{ ROI\_LR }\SpecialCharTok{+}\NormalTok{ (}\DecValTok{1}\SpecialCharTok{|}\NormalTok{ROI\_bilat\_name) }\SpecialCharTok{+}\NormalTok{ (}\DecValTok{1}\SpecialCharTok{|}\NormalTok{extracted\_run\_number) }\SpecialCharTok{+}\NormalTok{ (}\DecValTok{1}\SpecialCharTok{|}\NormalTok{subjectID)}
\NormalTok{    )}
\CommentTok{\# summary(rightlat\_domain\_model)}
\NormalTok{rightlat\_domain\_model }\SpecialCharTok{\%\textgreater{}\%}
  \FunctionTok{apa\_print}\NormalTok{() }\SpecialCharTok{\%\textgreater{}\%}
  \FunctionTok{apa\_table}\NormalTok{()}
\end{Highlighting}
\end{Shaded}

\begin{table}[tbp]

\begin{center}
\begin{threeparttable}

\caption{\label{tab:unnamed-chunk-22}}

\begin{tabular}{llllll}
\toprule
Term & \multicolumn{1}{c}{$\hat{\beta}$} & \multicolumn{1}{c}{95\% CI} & \multicolumn{1}{c}{$t$} & \multicolumn{1}{c}{$\mathit{df}$} & \multicolumn{1}{c}{$p$}\\
\midrule
Intercept & 2.64 & {}[1.94, 3.34] & 7.37 & 6.98 & < .001\\
Domain1 & 0.23 & {}[0.16, 0.31] & 6.10 & 2,009.00 & < .001\\
ROI LR1 & -0.33 & {}[-0.41, -0.26] & -8.63 & 2,009.00 & < .001\\
Domain1 $\times$ ROI LR1 & -0.21 & {}[-0.29, -0.14] & -5.56 & 2,009.00 & < .001\\
\bottomrule
\end{tabular}

\end{threeparttable}
\end{center}

\end{table}

\begin{Shaded}
\begin{Highlighting}[]
\FunctionTok{plot}\NormalTok{(}\FunctionTok{allEffects}\NormalTok{(rightlat\_domain\_model))}
\end{Highlighting}
\end{Shaded}

\includegraphics[width=0.5\linewidth]{1_Univariate_files/figure-latex/unnamed-chunk-22-1}

\begin{Shaded}
\begin{Highlighting}[]
\CommentTok{\# check\_model(rightlat\_domain\_model)}
\FunctionTok{check\_outliers}\NormalTok{(rightlat\_domain\_model)}
\end{Highlighting}
\end{Shaded}

\begin{verbatim}
## OK: No outliers detected.
\end{verbatim}

\begin{Shaded}
\begin{Highlighting}[]
\FunctionTok{lsmeans}\NormalTok{(rightlat\_domain\_model, pairwise }\SpecialCharTok{\textasciitilde{}}\NormalTok{ domain }\SpecialCharTok{|}\NormalTok{ ROI\_LR)}\SpecialCharTok{$}\NormalTok{contrasts}
\end{Highlighting}
\end{Shaded}

\begin{verbatim}
## ROI_LR = L:
##  contrast             estimate    SE   df t.ratio p.value
##  physics - psychology    0.042 0.108 2009 0.380   0.7010 
## 
## ROI_LR = R:
##  contrast             estimate    SE   df t.ratio p.value
##  physics - psychology    0.894 0.108 2009 8.240   <.0001 
## 
## Degrees-of-freedom method: kenward-roger
\end{verbatim}

\begin{Shaded}
\begin{Highlighting}[]
\NormalTok{rightlat\_domain\_model\_simpler }\OtherTok{\textless{}{-}} \FunctionTok{lmer}\NormalTok{(}
      \AttributeTok{data =}\NormalTok{ ROI\_data\_nonfocal\_runs12\_bilat }\SpecialCharTok{\%\textgreater{}\%}
        \FunctionTok{filter}\NormalTok{(ROI\_bilat\_name }\SpecialCharTok{\%in\%}\NormalTok{ right\_lat\_domain\_regions),}
      \AttributeTok{formula =}\NormalTok{ meanbeta }\SpecialCharTok{\textasciitilde{}}\NormalTok{ event }\SpecialCharTok{+}\NormalTok{ domain }\SpecialCharTok{+}\NormalTok{ (}\DecValTok{1}\SpecialCharTok{|}\NormalTok{extracted\_run\_number) }\SpecialCharTok{+}\NormalTok{ (}\DecValTok{1}\SpecialCharTok{|}\NormalTok{subjectID)}
\NormalTok{    )}
\FunctionTok{anova}\NormalTok{(rightlat\_domain\_model\_simpler, rightlat\_domain\_model)}
\end{Highlighting}
\end{Shaded}

\begin{verbatim}
## refitting model(s) with ML (instead of REML)
\end{verbatim}

\begin{verbatim}
## Data: ROI_data_nonfocal_runs12_bilat %>% filter(ROI_bilat_name %in%  ...
## Models:
## rightlat_domain_model_simpler: meanbeta ~ event + domain + (1 | extracted_run_number) + (1 | 
## rightlat_domain_model_simpler:     subjectID)
## rightlat_domain_model: meanbeta ~ domain * ROI_LR + (1 | ROI_bilat_name) + (1 | extracted_run_number) + 
## rightlat_domain_model:     (1 | subjectID)
##                               npar  AIC  BIC logLik deviance Chisq Df
## rightlat_domain_model_simpler    6 8431 8464  -4209     8419         
## rightlat_domain_model            8 8205 8250  -4095     8189   229  2
##                               Pr(>Chisq)    
## rightlat_domain_model_simpler               
## rightlat_domain_model             <2e-16 ***
## ---
## Signif. codes:  0 '***' 0.001 '**' 0.01 '*' 0.05 '.' 0.1 ' ' 1
\end{verbatim}

Below we test whether the following four regions that responded more to
unexpected than expected events on the right hemisphere, but not the
left (or only marginally on the left), show evidence for lateraliztion
(i.e.~a hemisphere x event effect).

\begin{Shaded}
\begin{Highlighting}[]
\NormalTok{right\_lat\_event\_regions }\OtherTok{\textless{}{-}} \FunctionTok{c}\NormalTok{(}\StringTok{"antParietal"}\NormalTok{, }
                       \StringTok{"midParietal"}\NormalTok{,}
                       \StringTok{"precentral\_A\_preCG"}\NormalTok{,}
                       \StringTok{"precentral\_B\_IFGop"}\NormalTok{,}
                       \StringTok{"supFrontal"}\NormalTok{)}

\NormalTok{rightlat\_event\_model }\OtherTok{\textless{}{-}} \FunctionTok{lmer}\NormalTok{(}
      \AttributeTok{data =}\NormalTok{ ROI\_data\_nonfocal\_runs12\_bilat }\SpecialCharTok{\%\textgreater{}\%}
        \FunctionTok{filter}\NormalTok{(ROI\_bilat\_name }\SpecialCharTok{\%in\%}\NormalTok{ right\_lat\_event\_regions),}
      \AttributeTok{formula =}\NormalTok{ meanbeta }\SpecialCharTok{\textasciitilde{}}\NormalTok{ event }\SpecialCharTok{*}\NormalTok{ ROI\_LR }\SpecialCharTok{+}\NormalTok{ (}\DecValTok{1}\SpecialCharTok{|}\NormalTok{ROI\_bilat\_name) }\SpecialCharTok{+}\NormalTok{ (}\DecValTok{1}\SpecialCharTok{|}\NormalTok{extracted\_run\_number) }\SpecialCharTok{+}\NormalTok{ (}\DecValTok{1}\SpecialCharTok{|}\NormalTok{subjectID)}
\NormalTok{    )}
\CommentTok{\# summary(rightlat\_event\_model)}
\NormalTok{rightlat\_event\_model }\SpecialCharTok{\%\textgreater{}\%}
  \FunctionTok{apa\_print}\NormalTok{() }\SpecialCharTok{\%\textgreater{}\%}
  \FunctionTok{apa\_table}\NormalTok{()}
\end{Highlighting}
\end{Shaded}

\begin{table}[tbp]

\begin{center}
\begin{threeparttable}

\caption{\label{tab:unnamed-chunk-23}}

\begin{tabular}{llllll}
\toprule
Term & \multicolumn{1}{c}{$\hat{\beta}$} & \multicolumn{1}{c}{95\% CI} & \multicolumn{1}{c}{$t$} & \multicolumn{1}{c}{$\mathit{df}$} & \multicolumn{1}{c}{$p$}\\
\midrule
Intercept & 3.32 & {}[2.35, 4.28] & 6.76 & 7.78 & < .001\\
Event1 & -0.18 & {}[-0.26, -0.10] & -4.43 & 2,520.00 & < .001\\
ROI LR1 & -0.47 & {}[-0.55, -0.39] & -11.43 & 2,520.00 & < .001\\
Event1 $\times$ ROI LR1 & 0.06 & {}[-0.02, 0.14] & 1.58 & 2,520.00 & .115\\
\bottomrule
\end{tabular}

\end{threeparttable}
\end{center}

\end{table}

\begin{Shaded}
\begin{Highlighting}[]
\FunctionTok{plot}\NormalTok{(}\FunctionTok{allEffects}\NormalTok{(rightlat\_event\_model))}
\end{Highlighting}
\end{Shaded}

\includegraphics[width=0.5\linewidth]{1_Univariate_files/figure-latex/unnamed-chunk-23-1}

\begin{Shaded}
\begin{Highlighting}[]
\CommentTok{\# check\_model(rightlat\_event\_model)}
\FunctionTok{check\_outliers}\NormalTok{(rightlat\_event\_model)}
\end{Highlighting}
\end{Shaded}

\begin{verbatim}
## OK: No outliers detected.
\end{verbatim}

\begin{Shaded}
\begin{Highlighting}[]
\FunctionTok{lsmeans}\NormalTok{(rightlat\_event\_model, pairwise }\SpecialCharTok{\textasciitilde{}}\NormalTok{ event }\SpecialCharTok{|}\NormalTok{ ROI\_LR)}\SpecialCharTok{$}\NormalTok{contrasts}
\end{Highlighting}
\end{Shaded}

\begin{verbatim}
## ROI_LR = L:
##  contrast    estimate    SE   df t.ratio p.value
##  exp - unexp   -0.232 0.115 2520 -2.020  0.0437 
## 
## ROI_LR = R:
##  contrast    estimate    SE   df t.ratio p.value
##  exp - unexp   -0.489 0.115 2520 -4.250  <.0001 
## 
## Degrees-of-freedom method: kenward-roger
\end{verbatim}

\begin{Shaded}
\begin{Highlighting}[]
\NormalTok{rightlat\_event\_model\_simpler }\OtherTok{\textless{}{-}} \FunctionTok{lmer}\NormalTok{(}
      \AttributeTok{data =}\NormalTok{ ROI\_data\_nonfocal\_runs12\_bilat }\SpecialCharTok{\%\textgreater{}\%}
        \FunctionTok{filter}\NormalTok{(ROI\_bilat\_name }\SpecialCharTok{\%in\%}\NormalTok{ right\_lat\_event\_regions),}
      \AttributeTok{formula =}\NormalTok{ meanbeta }\SpecialCharTok{\textasciitilde{}}\NormalTok{ event }\SpecialCharTok{+}\NormalTok{ event }\SpecialCharTok{+}\NormalTok{ (}\DecValTok{1}\SpecialCharTok{|}\NormalTok{extracted\_run\_number) }\SpecialCharTok{+}\NormalTok{ (}\DecValTok{1}\SpecialCharTok{|}\NormalTok{subjectID)}
\NormalTok{    )}
\FunctionTok{anova}\NormalTok{(rightlat\_event\_model\_simpler, rightlat\_event\_model)}
\end{Highlighting}
\end{Shaded}

\begin{verbatim}
## refitting model(s) with ML (instead of REML)
\end{verbatim}

\begin{verbatim}
## Data: ROI_data_nonfocal_runs12_bilat %>% filter(ROI_bilat_name %in%  ...
## Models:
## rightlat_event_model_simpler: meanbeta ~ event + event + (1 | extracted_run_number) + (1 | 
## rightlat_event_model_simpler:     subjectID)
## rightlat_event_model: meanbeta ~ event * ROI_LR + (1 | ROI_bilat_name) + (1 | extracted_run_number) + 
## rightlat_event_model:     (1 | subjectID)
##                              npar   AIC   BIC logLik deviance Chisq Df
## rightlat_event_model_simpler    5 11585 11614  -5787    11575         
## rightlat_event_model            8 11120 11167  -5552    11104   471  3
##                              Pr(>Chisq)    
## rightlat_event_model_simpler               
## rightlat_event_model             <2e-16 ***
## ---
## Signif. codes:  0 '***' 0.001 '**' 0.01 '*' 0.05 '.' 0.1 ' ' 1
\end{verbatim}

The main findings from this analysis were:

\hypertarget{domain-effects-1}{%
\subsection{Domain effects}\label{domain-effects-1}}

We again saw evidence that some MD regions show a univariate preference
for physical events over psychological events. The following MD regions
showed a main effect for domain: - Bilateral anterior parietal cortices
(\texttt{antParietal\_L} and \texttt{antParietal\_R}) with a marginal
effect on the left. - Bilateral midparietal cortices
(\texttt{midParietal\_L} and \texttt{midParietal\_R}) with a marginal
effect on the left - Bilateral posterior parietal cortices
(\texttt{postParietal\_L} and \texttt{postParietal\_R}) - Right
precentral cortices (precentral gyrus and IFG,
\texttt{precentral\_A\_preCG\_R} and \texttt{precentral\_B\_IFGop\_R}) -
Right superior frontal cortices (\texttt{supFrontal\_R}, superior to
both precentral regions) - Right midfrontal (\texttt{midFrontal\_R},
anterior to the precentral regions)

Notably, however, we did not see this across all MD regions. Regions
anterior to the sensorimotor strip did not show this effect (e.g.~no
domain effect emerged in insula, medial frontal or dACC, mid frontal
orbital or anterior LPFC). And in the 4 precentral regions showing
effect on the right, we found evidence for right lateralization of these
effects (side X domain interaction), with these four regions showing a
robust preference for physical than psychological events on the right,
but not the left.

We again see that the two visual regions (MT and V1, this time separated
into left and right hemispheres) respond more to physical than
psychological events.

The remaining regions were chosen to be putatively domain-specific, on
the basis of our prior findings as well as prior literature. We found
that indeed, these regions showed domain preferences in this experiment.
The following regions, that responded more to Heider + Simmel events
involving social than physical interaction showed a greater response, on
average, to psychological than physical stimuli.

\begin{itemize}
\tightlist
\item
  Bilateral lateral and ventral visual cortices
  (\texttt{vislateralventral\_L} and \texttt{vislateralventral\_R})
\item
  Right parahippocampal gyrus (\texttt{parahip-gyrus\_R})
\item
  Left superior inferior frontal cortices (\texttt{supinffrontal\_L})
\item
  Left and right medial prefrontal cortices (\texttt{MPFC\_L} and
  \texttt{MPFC\_R}),
\item
  (Already reported above in confirmatory analyses) left and right STS
  (\texttt{superiortemporal\_L} and \texttt{superiortemporal\_R})
\end{itemize}

And the following regions, that responded more to Heider + Simmel events
involving physical than social interaction, showed a greater response,
on average to physical than psychological stimuli.

\begin{itemize}
\tightlist
\item
  Bilateral medial visual cortices (\texttt{vismedial\_L} and
  \texttt{vismedial\_R})
\item
  Left and right superior parietal (\texttt{supparietal\_L} and
  \texttt{supparietal\_R})
\item
  Right precentral/superior frontal (\texttt{precentral-supfrontal\_R})
\item
  (Already reported above in confirmatory analyses) left and right SMG
  (\texttt{supramarginal\_L} and \texttt{supramarginal\_R})
\end{itemize}

\hypertarget{event-effects-1}{%
\subsection{Event effects}\label{event-effects-1}}

The following MD regions, selected to respond more when people are
engaging in difficult vs easy versions of a spatial working memory tasks
showed a main effect for event:

\begin{itemize}
\tightlist
\item
  Bilateral anterior parietal cortices (\texttt{antParietal\_L} and
  \texttt{antParietal\_R}, with a marginal effect on the left)
\item
  Right middle parietal cortex (\texttt{midParietal\_R})
\item
  Bilateral precentral gyrus (\texttt{precentral\_A\_preCG\_L} and
  \texttt{precentral\_A\_preCG\_R}, with a marginal effect on the left)
\item
  Right precentral IFG (\texttt{precentral\_B\_IFGop\_R})
\item
  Right superior frontal (\texttt{supFrontal\_R})
\end{itemize}

In contrast to the domain effects, we did not find robust evidence for
right lateralization, in spite of these effects emerging in right
regions and not left regions when analyzed separately.

We also found a marginal effect of event in right parahippocampal gyrus,
selected to respond more to social interaction
(\texttt{parahip-gyrus\_R}).

Notably we did not see any effects in MT or V1

\hypertarget{domain-specific-prediction-error-1}{%
\subsection{Domain-specific prediction
error}\label{domain-specific-prediction-error-1}}

We next looked for evidence for domain-specific prediction error, i.e.~a
domain x event interaction, in these regions.

In addition to an effect in left SMG, already reported above in the
confirmatory analysis, two additional regions showed this effect, both
marginal or near-marginal:

\begin{itemize}
\tightlist
\item
  Marginal effect in right medial frontal cortex
  (\texttt{medialFrontal\_R}), selected using the MD localizer, around
  dACC on the medial surface.
\item
  Near-marginal effect in right midfrontal cortex
  (\texttt{midFrontal\_R}), selected using the MD localizer, anterior to
  precentral MD regions, on the lateral surface.
\end{itemize}

Curiously, the midfrontal region showed a univariate preference for
physical events overall, but this region showed a stronger VOE effect
for psychological than physical events.

Even though both MT and V1 show a univariate preference for physical
events, there was no evidence that these visual regions respond to
unexpected events in either domain.

\hypertarget{part-3-both-domains}{%
\section{PART 3: Both domains}\label{part-3-both-domains}}

TBA

\end{document}
